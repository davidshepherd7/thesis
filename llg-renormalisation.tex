
% ??ds caption consistency with ODE LLG section, but make captions "better" as per comments
%% ??ds define the errors properly Refer to ODE LLG section.
\FloatBarrier
\chapter{Alternative magnetisation re-normalisation methods}
\label{sec:magn-renorm-meth}
\chaptermark{Alternative re-normalisation}

In this \thisref{sec:magn-renorm-meth} we experimentally compare the simplest method of ensuring constant $\abs{\mv}$: re-normalisation after every time step with the methods used in \nmag and \magpar.
\nmag uses the self-correcting LLG, described in \cref{sec:sc-llg}.
The \magpar package uses re-normalisation, but only re-normalises after the maximum error in the magnetisation length reaches a certain tolerance.

As our example problem we use the relaxing nano-sphere as described in \cref{sec:imr-ode-llg-numer-exper}.


\section{Implementation details}

As in \cref{sec:impl-deta-ode-llg} we use the Landau-Lifshitz form of the LLG.
The Newton-Raphson method is used for linearisation with the Jacobian calculated analytically and solved using a direct solver.
The Newton-Raphson tolerance set to $\ntol = 10^{-8}$.
We use BDF2 for these experiments because we need a time integrator with no geometric integration properties in order to fully test the re-normalisation methods (TR is equivalent to IMR for the undamped ODE LLG problem so it has the same geometric integration properties in this case, see \cref{sec:aimr-llgode-numerical-results}).
The time step size is fixed at $\dtn = 0.1$.

Re-normalisation after a tolerance is implemented as follows:
if the error in the magnetisation length, $\errml = \abs{1 - \abs{\mv}}$, is less than the tolerance, $\mltol$, then nothing is done and the integration continues.
On the other hand if $\errml > \mltol$ then we set $\mv_{i}$ to $\mv_{i}/\abs{\mv_{i}}$ for the values of the magnetisation at all times stored by the time integrator (for BDF2 this corresponds to $i=n+1, n, n-1$).
Note that re-normalising the history values of TR is more complex because one of them is a derivative (at least in our implementation, see \cref{eq:impl-tr}), which would need to be recalculated using the newly re-normalised magnetisation values for consistency.

We use tolerance values of $\mltol = 0, 10^{-6}, 10^{-2}, 10^{200}$.
Note that $\mltol \gg 1$ corresponds to no re-normalisation, while $\mltol = 0$ corresponds to re-normalisation after every time step.
The default value used in \magpar\footnote{Based on the source code of version 0.9, the relevant function is \texttt{CheckIterationLL\_Init}.} is $\mltol = 10^{-2}$.

The self correcting LLG is implemented by replacing the standard residual, $\rv_{\mathrm{ll}}$ given in \cref{eq:r-ode-llg}, with the modified residual
\begin{equation}
  \rv = \rv_{\mathrm{ll}}  - \scc \mv \bigb{1 - \abs{\mv}^2}.
\end{equation}
The Jacobian of the modified residual is given by
\begin{equation}
  \label{eq:J-ode-sc-llg}
  \Jm = \Jm_{\mathrm{ll}} + 2\scc \bigb{\mv \tensorprod \mv} - \scc \bigb{1 - \abs{\mv}^2} \Idm ,
\end{equation}
where $\Jm_{\mathrm{ll}}$ is given in \cref{eq:J-ode-llg}.

We experiment with a range of parameter values: $\scc = 0, 0.1, 1, 10, 100, 1000$.
Larger values were not used because the Newton-Raphson method often failed to converge within ten steps for $\scc \gtrsim 1000$.

Note that, due to implementation details, values are output \emph{after} any re-normalisation for that step has taken place.


\section{Results: re-normalisation after a tolerance}
\label{sec:renorm-after-toler}

\begin{figure}
  \centering
  \includegraphics[width=0.8\textwidth]{{{plots/tolrenorm-geom-properties/0.01-mlengtherrormaxesvstimes}}}
  \caption{
    Error in magnetisation length against time
    for the ODE LLG problem
    with $\dampc = 0.01$.
    The legend indicates the values of $\mltol$, the tolerance on the magnetisation length after which the magnetisation is re-normalised.
  }
  \label{fig:renorm-tol-ml-err}
\end{figure}

\begin{figure}
  \centering
  \includegraphics[width=0.8\textwidth]{{{plots//tolrenorm-geom-properties/0-mlengtherrormaxesvstimes}}}
  \caption{
    Error in magnetisation length against time
    for the ODE LLG problem with
    $\dampc = 0$.
    The legend indicates the values of $\mltol$, the tolerance after which the magnetisation is re-normalised.
  }
  \label{fig:renorm-tol-ml-err-undamped}
\end{figure}

In \cref{fig:renorm-tol-ml-err,fig:renorm-tol-ml-err-undamped} we show the error in the magnetisation length over time
\begin{equation}
  \errml = \abs{\abs{\mv} - 1}
\end{equation}
for various values of $\mltol$ and $\dampc=0.01, 0$ respectively.
The small error values for the damped case with $\mltol = 10^{-6}$ at $t \gtrsim 280$ are caused by the magnetisation being re-normalised after (almost) every step.
In both the damped and undamped cases, the use of the intermediate tolerance value, $\mltol = 10^{-6}$, results in oscillations of the magnetisation length.
For $\mltol = 10^{-2}$ in the undamped case the error in the magnetisation length never reaches the tolerance and the behaviour is identical to no re-normalisation.
In the damped case the tolerance $\mltol = 10^{-2}$ is reached and the behaviour is similar to that of $\mltol = 10^{-6}$, except that the oscillations begin at a later time.


\begin{figure}
  \centering
  \includegraphics[width=0.8\textwidth]{plots/tolrenorm-geom-properties/0-absofenergychangevstimes.pdf}
  \caption{
    Error in energy against time
    for the ODE LLG problem
    with $\dampc = 0$.
    The legend indicates the values of $\mltol$, the tolerance after which the magnetisation is re-normalised.
  }
  \label{fig:renorm-tol-energy-err}
\end{figure}

In \cref{fig:renorm-tol-energy-err} the errors in the energy for the undamped case are shown.
As noted previously, $\mltol= 10^{-2}$ behaves as the non-renormalised case for this example.
When $\mltol = 10^{-6}$ we see an error similar to that for the always re-normalised case, except with additional oscillations in the energy corresponding to times where a re-normalisation is carried out.


\begin{figure}
  \centering
  \includegraphics[width=0.8\textwidth]{plots/tolrenorm_llg_ode_convergence/maxofswitchingtimeerrorvsmeanofdts}
  \caption{
    Convergence of the maximum (over all steps) of the error in the switching time
    against step size
    for the ODE LLG problem with
    $\dampc = 0.01$.
    The legend indicates the tolerance after which the magnetisation is re-normalised.
    ??ds not actually a mean...
  }
  \label{fig:tol-renorm-convergence}
\end{figure}

Finally in \cref{fig:tol-renorm-convergence} we show the convergence of the method (in terms of the error in the switching time) as the step size is reduced for the damped case.
The use of $\mltol=10^{-2}$ or no re-normalisation ($\mltol=10^{200}$) gives slightly worse errors than the tighter tolerances.


\FloatBarrier
\section{Results: the self-correcting LLG}
\label{sec:self-correcting-llg-results}

\begin{figure}
  \centering
  \includegraphics[width=0.8\textwidth]{{{plots/sc-geom-properties/0.01-mlengtherrormaxesvstimes}}}
  \caption{
    Error in magnetisation length against time
    for the ODE self-correcting LLG problem
    with $\dampc = 0.01$.
    The legend indicates the values of $\scc$.
  }
  \label{fig:sc-ml-err}
\end{figure}

\begin{figure}
  \centering
  \includegraphics[width=0.8\textwidth]{plots//sc-geom-properties/0-mlengtherrormaxesvstimes.pdf}
  \caption{
    Error in magnetisation length against time
    for the ODE self-correcting LLG problem
    with $\dampc = 0$.
    The legend indicates the values of $\scc$.
  }
  \label{fig:sc-ml-err-undamped}
\end{figure}

In \cref{fig:sc-ml-err,fig:sc-ml-err-undamped} we show the errors in magnetisation length over time with various values of the parameter $\scc$ for the damped and undamped problems respectively.
As would be expected larger values of $\scc$ reduce the error in both cases.
%??ds what's up with that initial transient...
Note that in the damped case (\cref{fig:sc-ml-err}) with $\scc = 1000$ the curve stops at $t \sim 480$, this is due to the failure of the Newton-Raphson method to converge within the ten steps allowed.

\begin{figure}
  \centering
  \includegraphics[width=0.8\textwidth]
  {{{plots/sc-geom-properties/0-absofenergychangevstimes}}}
  \caption{
    Error in energy against time
    for the ODE self-correcting LLG problem
    with $\dampc = 0$.
    The legend indicates the values of $\scc$.
  }
  \label{fig:sc-energy-err}
\end{figure}

In \cref{fig:sc-energy-err} we show the error in the energy for the undamped ODE LLG problem with various values of $\scc$.
We see that any $\scc > 0$ causes an error in the energy similar to that caused by the various re-normalisation approaches shown in \cref{fig:renorm-tol-energy-err}.


\begin{figure}
  \centering
  \includegraphics[width=0.8\textwidth]
  {{{plots/self_correcting_llg_ode_convergence/maxofswitchingtimeerrorvsmeanofdts}}}
  \caption{
    Convergence of the maximum (over all steps) of the error in the switching time
    against step size
    for the ODE self-correcting LLG problem with
    $\dampc = 0.01$.
    The legend indicates, respectively, the values of $\scc$ and the tolerance after which the magnetisation is re-normalised.
  }
  \label{fig:sc-convergence}
\end{figure}

In \cref{fig:sc-convergence} we show the convergence of the maximum error in the switching time against the time step size $\dtn$ with two values of $\scc$.
For comparison we also show the results when re-normalisation after every step is used instead of the self correcting term (\ie $\mltol = 0$ and $\scc = 0$).
We see that the resulting errors are essentially identical for all three cases.

\begin{table}
  \begin{subtable}{.5\textwidth}
    \centering
    \begin{tabular}{ll}
      $\scc$ & Mean Newton iterations \\
      \hline
      0 & 2.0 \\
      0.1 & 2.0 \\
      1 & 2.46 \\
      10 & 2.68 \\
      100 & 3.13 \\
      1000 & 3.12* \\
    \end{tabular}%
    \caption{Solved with BDF2 and $\ntol = 10^{-8}$}
  \end{subtable}%
  \begin{subtable}{.5\textwidth}
    \centering
    \begin{tabular}{ll}
      $\scc$ & Mean Newton iterations \\
      \hline
      0 & 2.0 \\
      0.1 & - \\
      1 & - \\
      10 & - \\
      100 & - \\
      1000 & - \\
    \end{tabular}%
    \vfill
    \caption{Solved with IMR and $\ntol = 10^{-12}$}
  \end{subtable}%
  \caption{
    Mean (over all time steps) number of Newton iterations for convergences
    for the ODE self-correcting LLG problem with
    $\dampc = 0.01$.
    The numerical methods used are indicated in the sub-captions.
    Values of $\scc \neq 0$ are not considered when using IMR because no self-correcting term is needed to conserve $\abs{\mv}$.
    Note that with $\scc = 1000$ the Newton-Raphson method fails to converge at $t \sim 480$.
  }
  \label{tab:sc-newton-iters}
\end{table}


Finally in \cref{tab:sc-newton-iters} we show the number of Newton-Raphson iterations required for convergence for each value of $\scc$.
For comparison we also show the result when using IMR with Newton tolerance $\ntol=10^{-12}$ (recall that a sharp linearisation tolerance is required for IMR to attain good geometric integration properties).
We see that for larger values of $\scc$ additional iterations are required.
In addition we see that the tighter tolerance required for geometric integration with IMR has no effect on the number of iterations.

Since $\scc \geq 100$ is required to control $\errml$ to even the fairly loose level of $10^{-6}$ the use of the self correcting LLG requires at least one additional Newton-Raphson iteration.
As the majority of the computation time for a time step is spent within the Newton-Raphson method, this increased the cost of each step by a factor of approximately $3/2$.
Since the use of the self correcting LLG does not provide any improvement in the overall accuracy over re-normalising after every step (so the same $\dtn$ is required to obtain the same accuracy), this increases the overall cost of the method by the same factor.


\section{Conclusions}

By design, the use of re-normalisation after a tolerance allows larger errors in $\abs{\mv}$ than re-normalisation after every step.
It also introduces spurious oscillations in $\abs{\mv}$ and in the error in the energy for the undamped case.
It has a slight negative effect on convergence when compared to re-normalisation after every step, and no effect on the computational cost of the re-normalisation.
Hence there is no reason to prefer re-normalisation after a tolerance over re-normalisation after every step.

The self correcting LLG is worse at controlling errors in $\abs{\mv}$ than re-normalisation after each step, and gives very similar errors in the energy.
It gives essentially identical convergence properties as re-normalisation after every step.
Finally it requires additional step(s) of the Newton-Raphson method convergence (typically one more step is needed, increasing the computational time for the simulation by a factor of $\sim 3/2$).
As such there is no reason to use the self-correcting LLG over re-normalisation after each step.


%%% Local Variables:
%%% mode: latex
%%% TeX-master: "main"
%%% End:
