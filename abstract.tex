Micromagnetics is a continuum mechanics theory of magnetic materials widely used in industry and academia.
In this thesis we describe a complete numerical method, with a number of novel components, for the computational solution of dynamic micromagnetic problems by solving the Landau-Lifshitz-Gilbert (LLG) equation.
In particular we focus on the use of the implicit midpoint rule (IMR), a time integration scheme which conserves several important properties of the LLG equation.
We use the finite element method for spatial discretisation, and use nodal quadrature schemes to retain the conservation properties of IMR despite the weak-form approach.

We introduce a novel, generally-applicable adaptive time step selection algorithm for the IMR.
The resulting scheme selects error-appropriate time steps for a variety of problems, including the semi-discretised LLG equation.
We also show that it retains the conservation properties of the fixed step IMR for the LLG equation.

We demonstrate how hybrid FEM/BEM magnetostatic calculations can be coupled to the LLG equation in a monolithic manner.
This allows the coupled solver to maintain all properties of the standard time integration scheme, in particular stability properties and the energy conservation property of IMR.
We also develop a preconditioned Krylov solver for the coupled system which can efficiently solve the monolithic system provided that an effective preconditioner for the LLG sub-problem is available.

Finally we investigate the effect of the spatial discretisation on the comparative effectiveness of implicit and explicit time integration schemes (\ie the stiffness).
We find that explicit methods are more efficient for simple problems, but for the fine spatial discretisations required in a number of more complex cases implicit schemes become more efficient.

%%% Local Variables:
%%% mode: latex
%%% TeX-master: "main"
%%% End:
