\chapter{Test problems}
\label{sec:test-problems}


Ideally test problems should have the following characteristics

\begin{itemize}
\item Contains effects from all effective fields
\item Has magnetisation variations in both time and space
\item Has known analytical solutions
\item Has solutions using other software for comparision
\end{itemize}

unfortunately there are no such problems. 
So instead we use a collection of different problems as appropriate.

\section{Problems with analytical solutions}

As far as the authors are aware there are no solutions to the LLG with both magnetostatics and exchange coupling and spatial variations in magnetisation. 
So again we need multiple problems.

\subsection{Switching of a single domain ellipsodial particle}

The first problem is a simple coherrent reversal of 


\subsection{Wave-like solution}

Only exchange effective field

Applicable to any number of dimensions

Damping allowed


\section{Problems with no analytical solution}

Both of the above problems are vastly simpler than the problems typical solved in micromagnetic models.

\subsection{The \mumag standard problem 4}

Widely used to test micromagnetic codes

Reversal of a thin film of permalloy under two different fields

Unfortunately FEM/BEM magnetostatic calcultations are extremely unsuited to thin film problems (the dense BEM matrix size is proportional to the size of the boundary, in thin films the entire problem is on the boundary. The additional geometric flexibility is not needed for simple cubeoid shapes.)
But we will do it anyway in order to test our results.

The problem specification is as follows:

The magnetic domain is a simple sheet of magnetic material $500 \times 125 \times 3$nm with material parameters
\begin{equation}
  \begin{aligned}
    A &= 1.3\E{-11} \text{J/m}, \\
    M_s &= 8.0\E{5} \text{A/m}, \\
    \Kone &= 0.0, \\
    \gymagc &= 2.211 \E{5} \text{m/As}, \\
    \dampc &= 0.02.
  \end{aligned}
\end{equation}
Two different applied fields should be used, correspond to two different solutions:
\begin{equation}
  \begin{aligned}
    \happ_1 = [-24.6, 4.3, 0.0] \E{-3}\text{A/m}, \\
    \happ_1 = [-35.5, -6.3, 0.0] \E{-3}\text{A/m}, \\
  \end{aligned}
\label{eq:mumag-h-app}
\end{equation}
where we have converted from the magnetic flux intensity specified by the \mumag website to magnetic field by dropping a factor of $\mu_0$ from the RHS.
The initial condition is the result of relaxing the magnetisation from the state created by a saturating field in the $[1,1,1]$.

The magnetic parameters result in a magnetostatic exchange length (and simulation unit length) of
\begin{equation}
  l_{\text{ex}} = \sqrt{\frac{2A}{\mu_0 M_s^2}} = 5.6858\text{nm},
\end{equation}
hence the normalised dimensions are $500 \times 125 \times 3 / 5.6858$.
The unit time is
\begin{equation}
  t_{\text{unit}} = \frac{1}{\gymagc M_s} = 5.653\text{ps}.
\end{equation}
The normalised applied fields are simply the fields given in \cref{eq:mumag-h-app} divided by $M_s = 8.0\E{5}$.


\subsection{Reversal of a cylinder}

More relevant problem for FEM/BEM methods--non-trivial geometry.






%%% Local Variables:
%%% mode: latex
%%% TeX-master: "main"
%%% End:
