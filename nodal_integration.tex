\section{Nodal integration}


In weak form methods it turns out that normal integration schemes do not retain the conservation properties of the IMR.
The problem is that (outside of the infinite node limit) weak form equations make statements about the integral properties of the solution, whereas magnetisation length is a nodal property.
The solution is to use a non-typical quadrature scheme which directly links the nodal values with the integral values.
The downside of such schemes is that the accuracy of the evaluation of integrals is reduced.

In the micromagnetics literature this is known as reduced integration\cite{Cimrak2008}.
However in other finite element literature ``reduced integration'' refers to using lower order gaussian quadrature than needed to exactly integrate the shape and test functions.
The term ``Nodal integration'' is more the standard term for schemes where the nodal values are used directly in the quadrature (typically with mesh-free methods \eg \cite{Puso2008}).

In this section we first show why standard quadrature schemes cannot have the required conservation properties.
We then show that the introduction of nodal integration schemes reattains the conservation properties.
Finally we present some numerical experiments.



\subsection{Failure of $\abs{\mv}$ conservation with Gaussian quadrature}

\newcommand{\ipg}[2]{\intd{{#1} \cdot {#2}}}

We begin with the weak form of the LLG equation
\begin{equation}
  \label{eq:weak-llg}
  \intd{\dmdt \cdot \testv + \mv \times \hv[t, \mv] \cdot \testv - \dampc \mv \times \dmdt \cdot \testv} = 0, \quad \forall \testv
\end{equation}
where $\testv = \threevec{\test}{\test}{\test}$ and $\test \in \ts$ is a scalar test function.

\newcommand{\midpoint}[1]{\hat{#1}}
\newsubcommand{\mvm}{\midpoint{\mv}}{n}
\newcommand{\tm}{\midpoint{t}_n}

\newcommand{\dtop}{\delta}
\newcommand{\dmdtm}{\dtop \mv_n}

% To simplify the notation we use inner product notation for the integral of a dot product:
% \begin{equation}
%   \intd{\av \cdot \bv} = \ipg{\av}{\bv}.
% \end{equation}

Substituting in the IMR we obtain
\begin{equation}
  \label{eq:weak-llg-imr}
  \ipg{\dmdtm}{\testv} + \ipg{(\mvm \times \hv[\tm, \mvm])}{\testv} - \dampc \ipg{(\mvm \times \dmdtm)}{\testv} = 0, \quad \forall \testv .
\end{equation}

where $\midpoint{x} = \frac{x_{n+1} + x_{n}}{2}$ is the midpoint value of $x$ and $\dtop x = \frac{x_{n+1} - x_n}{\dtn}$ is the midpoint derivative.

To obtain our result we examine the case where $\testv = \mvm$.
Note that $\mvm$ is in the vector space of test functions because we are using identical test and shape function spaces and $\mvm$ at any point must only be a linear combination of shape functions.
So by \eqref{eq:weak-llg-imr}, and using the fact that $(\av \times \bv) \cdot \av) = 0$ we have
\begin{equation}
\label{eq:23}
  \begin{aligned}
    0 &= \ipg{\dmdtm}{\mvm}, \\
    &= \frac{1}{2\dtn} \intd{(\mv_{n+1} + \mv_{n}) \cdot (\mv_{n+1} - \mv_n)}, \\
    &= \frac{1}{2\dtn} \intd{\abs{\mv_{n+1}}^2 - \abs{\mv_{n}}^2}.
  \end{aligned}
\end{equation}

At first glance it appears that we have achieved conservation. However this is only an integral relationship, meaning that the values of the integrand at the nodes are not constrained.
We can see this in more detail by substituting in the space interpolation of $\mv$ at the Gaussian integral evaluation points.
Dropping the constant factor of $2\dtn$ and assuming that $\abs{\mv_n} = 1$ everywhere this gives
\begin{equation}
  \begin{aligned} 
    1 &= \intd{\abs{\sum_k \mv_{n+1, k} \tbf_k(\xv)}^2}, \\
    &= \sum_l w_l \abs{\sum_k \mv_{n+1, k} \tbf_k(\xv_l)}^2.
  \end{aligned} 
\end{equation}

This can be satisfied without requiring that all $\abs{\mv_{n+1, k}} = 1$.
For example if we set the number of nodes to two (a 1D problem with linear shape functions) and assume magnetisation only along the $z$-axis then this condition becomes:
\begin{equation}
  \begin{aligned}
    1 &= \sum_l w_l \abs{\sum_k \mv_{n+1, k} \tbf_k(\xv_l)}^2, \\
    &= w_0 (m_0 a + m_1 b)(m_0 a + m_1 b) + w_1 (m_0 c + m_1 d)(m_0 c + m_1 d), \\
    &= m_0^2 (w_0 a^2 + w_1 c^2) + m_1^2 (w_0 b^2 + w_1 c^2) + m_0 m_1 (2w_0ab + 2w_1cd), \\
    &= m_0^2 \alpha + m_1^2 \beta + m_0 m_1 \gamma, \\
  \end{aligned}
\end{equation}
where $a,b,c,d$ are the values of shape functions at the integration points, $m_l = m_{z}$ at node l and $\alpha, \beta, \gamma$ are constants.
So given any $m_{z,0}$ we can solve the above expression to find an $m_{z,1}$ that satisfies the constraint.
Since $m_z$ all magnetisation is along the $z$-axis in this example we can vary magnetisation length arbitraily while still satisfying the constraint.

A similar expression is obtained if nodal magnetistations are chosen for the test function used in \eqref{eq:23} instead of the magnetisation function, as in \autoref{sec:weak-cons-absmv}.

Finally it is interesting to note that if the magnetisation is constant in space at times $t_n$ and $t_{n+1}$ then the integral condition in \eqref{eq:23} \emph{is} sufficient to give conservation of the magnetisation length at nodes.


\subsection{Failure of energy conservation with Gaussian quadrature}

The proof of energy conservation does not fail directly.
However, non-constant magnetisation length means that the derivation of the boundary condition is no longer applicable (??ds check this, non-constant |m| should destroy neumann condition therefore symmetry is gone).
Hence the effective field is no longer symmetrical and the proof of energy conservation no longer applies.

\subsection{Nodal integration}

In order to regain conservation properties in a weak-form-based method we introduce a nodal integration scheme based on that used by Bartels et. al.\cite{Bartels2006}:
\begin{equation}
  \label{eq:nodal-integration}
  \int f(\xv) \d \xv \approx \sum_{l\in \text{nodes}} \beta_l f(\xv_l),
\end{equation}
where $\beta_l$ is a weight.
This is simply the weighted sum of the value of the integrand at nodes.

As an additional benefit this greatly simplifies the calculations since no interpolation of the values to the integration points is needed.
For example with reduced integration (and using that $\tbf_k(\xv_l) = \delta_{kl}$) the residual contribution of the time derivative term of LLG of test function $k$ becomes
\begin{equation}
  \begin{aligned}
    \intd{\dmdt \cdot \tbfv_k} &= \frac{1}{\dtn} \intd{(\mv_{n+1}(\xv) + \mv_{n}(\xv)) \cdot \tbfv_k(\xv)}, \\
    & = \frac{1}{\dtn} \sum_{l\in \text{nodes}} \beta_l (\mv_{n+1, l} + \mv_{n, l}) \cdot \tbfv_k(\xv_l), \\
    & = \frac{1}{\dtn} \beta_k (\mv_{n+1, k} + \mv_{n, k}) \cdot \threevec{1}{1}{1}.
  \end{aligned}
\end{equation}


\subsubsection{Derivation of weights}

We now need to derive a suitable $\beta_l$.
To do so we represent the quadrature scheme as an integral of an interpolating polynomial which matches the desired integration, then we rearrange the equation to find the weights \cite[pg. 480]{Kincaid2002}.
In \texttt{oomph-lib} all integration is done in local co-ordinates (as discussed in \autoref{sec:fem-integration-??ds}), so we calculate weights applicable in this case.

A suitable polynomial for our case is the shape/test function for the node.
So we have:
\begin{equation}
\label{eq:nodal-quad-weights}
  \begin{aligned}
    \int_{\magd_e} f(\xv) \d \xv &= \int_{\magd_e} f(\sv) \pd{\sv}{\xv} \d \sv, \\
    &\approx \int_{\magd_e} \sum_l f(\sv_l) \tbf_l(\sv) J_l \d \sv, \\
    &\approx  \sum_l f(\sv_l) J_l \int_{\magd_e} \tbf_l(\sv)  \d \sv,
  \end{aligned} 
\end{equation}
where $J_l =  \evalat{\pd{\sv}{\xv}}_{\sv=\sv_l}$ is the Jacobian of the transformation from local to global coordinates evaluated at the $l$-th integration point.
So 
Comparing \eqref{eq:nodal-integration} with \eqref{eq:nodal-quad-weights} we see that
\begin{equation}
  \beta_l =  \int_{\magd_e} \tbf_l(\sv)  \d \sv.
\end{equation}

The equivalent weight for global integration is $\beta_l = \int_{\magd} \tbf_l(\xv) \d \xv$, as used by Bartels \etal\cite{Bartels2006}.

\subsection{Conservation of $\abs{\mv}$ at nodes}
\label{sec:weak-cons-absmv}

Starting from the IMR discretised weak form of the LLG \eqref{eq:weak-llg-imr}:
\begin{equation}
  \intd{\dmdtm \cdot \testv + (\mvm \times \hv[\tm, \mvm]) \cdot \testv - \dampc (\mvm \times \dmdtm) \cdot \testv }= 0, \quad \forall \testv .
\end{equation}

\newcommand{\dmdtml}{\dtop \mv_{n,l}}
\newcommand{\dmdtmj}{\dtop \mv_{n,j}}

We examine the choice of the test function as the midpoint nodal value of $\mv$ at node j multiplied by the $j$th test basis function, $\testv = \mvm_{,j} \tbf_j$.
This choice is clearly in the vector space of test functions because it is simply a constant multiple of a basis function.
\begin{equation}
  \sum_l \beta_l \bigs{\dmdtml \cdot \mvm_{,j}\tbf_j(\xv_l) + (\mvm_{,l} \times \hv[\tm, \mvm_{,l}]) \cdot \mvm_{,j}\tbf_j(\xv_l) - \dampc (\mvm_{,l} \times \dmdtml) \cdot \mvm_{,j}\tbf_j(\xv_l)} = 0.
\end{equation}

Using $\tbf_k(\xv_l) = \delta_{kl}$ we can eliminate the summation
\begin{equation}
  \beta_j \bigs{\dmdtmj \cdot \mvm_{,j} + (\mvm_{,j} \times \hv[\tm, \mvm_{,j}]) \cdot \mvm_{,j} - \dampc (\mvm_{,j} \times \dmdtmj) \cdot \mvm_{,j}} = 0.
\end{equation}
By the properties of the triple product the procession and damping terms vanish. Then expanding the midpoint values gives us our conservation result:
\begin{equation}
  \begin{aligned}
    \frac{\beta_l}{2\dtn}(\mv_{n+1,j} - \mv_{n,j}) \cdot (\mv_{n+1, j} + \mv_{n, j}) &= 0, \\
    \abs{\mv_{n+1, j}}^2 - \abs{\mv_{n, j}}^2 &= 0.
  \end{aligned}
\end{equation}

This can obviously be repeated for all nodes $j$, hence each nodal magnetisation length is conserved.


\subsection{Energy loss}

Again we start from the IMR discretised weak form of the LLG \eqref{eq:weak-llg-imr} and proceed by choosing specific test functions.
In this case we first choose $\testv = \dmdtm$, which gives:
\begin{equation}
  \label{eq:test-dmdt}
  \begin{aligned}
    0 &= \intd{\dmdtm \cdot \dmdtm} - \intd{\dmdtm  \cdot (\mvm \times \hv[\tm, \mvm])}.
  \end{aligned}
\end{equation}

Secondly we choose $\testv = \hv[\tm, \mvm]$ to obtain:
\begin{equation}
  \label{eq:test-h}
  \begin{aligned}
    0 &= \intd{\dmdtm \cdot \hv[\tm, \mvm]} - \dampc \intd{\hv[\tm, \mvm] \cdot (\mvm \times \dmdtm)}, \\
    & = \intd{\dmdtm \cdot \hv[\tm, \mvm]} - \dampc \intd{\dmdtm \cdot (\hv[\tm, \mvm] \times \mvm)}, \\
    & = \intd{\dmdtm \cdot \hv[\tm, \mvm]} + \dampc \intd{\dmdtm \cdot (\mvm \times \hv[\tm, \mvm])}.
  \end{aligned}
\end{equation}

Combining equations~\eqref{eq:test-dmdt} and \eqref{eq:test-dmdt} results in
\begin{equation}
  0 = \intd{\dmdtm \cdot \hv[\tm, \mvm]} + \dampc \intd{\dmdtm \cdot \dmdtm},
\end{equation}
\ie
\begin{equation}
  \intd{\mv_{n+1} \cdot \hv[\tm, \mvm]} = \intd{\mv_n \cdot \hv[\tm, \mvm]} - \dtn \dampc \intd{\dmdtm \cdot \dmdtm}.
\end{equation}
With zero applied field $\hv[t, \mv] = \hv[\mv]$ is a symmetrical linear operator on $\mv$ with respect to the inner product $\intd{\av \cdot \bv}$ (see \autoref{sec:energy-field-relation}), using this property in a similar manner to \autoref{sec:prop-imr-llg} we obtain
\begin{equation}
  \frac{1}{2}\intd{\mv_{n+1} \cdot \hv[\mv_{n+1}]} = \frac{1}{2}\intd{\mv_n \cdot \hv[\mv_n]} - \dtn \dampc \intd{\dmdtm \cdot \dmdtm}.
\end{equation}
But, as mentioned in \autoref{sec:prop-imr-llg} $\frac{1}{2}\intd{\mv \cdot \hv[\mv]}$ is the total micromagnetic energy of the domain (when $\happ = \zerov$), hence we have
\begin{equation}
  \e_{n+1} = \e_n - \dtn \dampc \intd{\dmdtm \cdot \dmdtm},
\end{equation}
which is the midpoint discretisation of the analytical energy loss rate given by equation~\eqref{eq:energy-decay}.
In particular when $\dampc = 0$ the energy is conserved.

The extension to the case with the applied field proceeds exactly as for the strong form of the LLG, as discussed in \autoref{sec:prop-imr-llg}.
The final result is
\begin{equation}
  ??ds
\end{equation}

\subsection{Numerical experiments with nodal integration schemes}


\subsection{Numerical experiments on $\abs{\mv}$ conservation}

We use the wave exact solution (see \autoref{wave-solution}) for simplicity.
Newton tolerance is set to $\E{-14}$.
Solved using GMRES with an ilu-1 preconditioner.

\begin{figure}[ht!]
  \centering
  \includegraphics[width=0.8\textwidth]{plots/2d_wave_solution_m_length/gauss-meanmathbfm-1vst.pdf}
  \caption{Evolution of the error of magnetisation length in the 2D wave example with a standard Gaussian quadrature scheme.}
  \label{fig:mean-ml-error-2d-gauss}
\end{figure}

\begin{figure}[ht!]
  \centering
  \includegraphics[width=0.8\textwidth]{plots/2d_wave_solution_m_length/lnodal-meanmathbfm-1vst.pdf}
  \caption{Evolution of the error of magnetisation length in the 2D wave example with the nodal quadrature scheme introduced above.}
  \label{fig:mean-ml-error-2d-nodal}
\end{figure}




%%% Local Variables:
%%% mode: latex
%%% TeX-master: "main"
%%% End:
