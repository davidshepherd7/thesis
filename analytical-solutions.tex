\section{Analytical solutions of the Landau-Lifshitz-Gilbert equation}

When testing numerical solvers for differential equations it is very
helpful to be able to check the results against analytical solutions for
some special cases. This section contains some solutions useful for this
purpose.
Unfortunately it appears to be very difficult to find solutions to the LLG equation with exchange coupling (\ie spatially dependant/spatially non-constant).

\subsection{Solution due to Mallinson}

In a 2000 paper\cite{Mallinson2000} Mallinson gives an analytical equation for the time taken for magnetisation to ``switch'' from one polar angle (angle to the field axis) to another.
He also gives the azimuthal angle (the angle around the field axis) rotated through during this switching.

The conditions for this model to apply are:
\begin{enumerate}
\item Constant applied field.
\item No exchange field.
\item Uniaxial anisotropy.
\item Magnetostatic field (prolate ellipsoidal particles only).
\item All effective fields must lie along the same axis.
\end{enumerate}

Let $\theta_1$, $\theta_2$ be the initial and final angles between the field axis and the magnetisation (\ie initial and final polar angles in the spherical polar coordinate system with the field axis as the main axis). Let $H_k$ be the combined anisotropy field: $H_k = \frac{2 K}{M_s} + M_s(N_\perp - N_\parallel)$ where $N$ is the demagnetisation tensor of the ellipsoid. All other symbols have their usual meanings. Then the time taken to switch from $\theta_1$ to $\theta_2$ is

\begin{equation}
  \tau = \frac{\dampc^2 +1}{\gymagc \dampc} \frac{1}{H^2 - H_k^2}
  \left[ H \ln \left( \frac{\tan(\theta_2/2)}{\tan(\theta_1/2)} \right)
       + H_k \ln \left( \frac{H - H_k \cos\theta_1}{H - H_k \cos\theta_2} \right)
       + H_k \ln \left( \frac{\sin\theta_2}{\sin\theta_1} \right)
    \right].
\end{equation}

The azimuthal angle precessed through during this switching is
\begin{equation}
  \phi = \frac{-1}{\dampc} \ln \left( \frac{\tan(\theta_2/2)}{\tan(\theta_1/2)} \right).
\end{equation}

Note that this is not really a true ``solution'' to the Landau-Lifshitz-Gilbert equation.
It gives the switching time and azimuthal angle as a function of polar angle rather than the magnetisation direction as a function of time. However comparing the exact values with those generated by a model is still a useful test.

\subsection{Solution due to Serpico et. al.}

Serpico et. al. \cite{Serpico2003} give a complete solution for the \emph{undamped} spatially constant Landau-Lifshitz equation.




\subsection{Constant field solution}

Another analytical solution based on the non-Gilbert form of the Landau-Lifshitz-Gilbert equation is given by Jiang et. al.\cite{Jiang2001}

\begin{align*}
  \phi(t) &= \gamma' H t, \\
  \cos \theta(t) &= \tanh (\gamma' \dampc H (t - t_0)), \\
  \sin \theta(t) &= \left[ \cosh (\gamma' \dampc H (t - t_0)) \right]^{-1}, \\
\end{align*}
where $\gamma' = \frac{\gymagc}{1 + \alpha^2}$ and
\begin{equation}
  t_0 = \frac{1}{\gamma' \dampc H} \log \left( \frac{\sin \theta_0}{1 + \cos \theta_0} \right)
\end{equation}


??ds note on use as a semi-analytical time integrator

\subsection{Stationary states of the LLG with exchange}

It is also fairly easy to construct stationary solutions to the Landau-Lifshitz-Gilbert equation with only the exchange field included.
The full equation for this case is
\begin{equation}
  \dmdt = - \left( \mv \times \lap \mv \right) + \dampc \left( \mv \times \dmdt \right).
\end{equation}

When $\dmdt = 0$, \ie a steady state, this reduces to
\begin{equation}
   \left( \mv \times \lap \mv \right) = 0.
\end{equation}

One case when this is true is
\begin{equation}
  \lap \mv \propto \mv,
  \label{eq:steadystatecond}
\end{equation}
(also when $\lap \mv = 0$ or $\mv = 0$ but these are not useful for testing).
We can construct an $\mv$ situation where \eqref{eq:steadystatecond} holds from combinations of, for example, $\cos(x)$ and $\sin(x)$.
An example of a complete solution is
\begin{equation}
  \mv = \left[
    \begin{array}{c}
      sin(x) + cos(y) \\ cos(x) + sin(y) \\ cos(z)
    \end{array}
    \right].
\end{equation}

Then $\lap \mv = - \mv$ as required.
