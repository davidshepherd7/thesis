
\chapter{Non-dimensionalisation}
\label{sec:normalisations-appendix}

%??ds talk about why?

% no unecessary (for the maths) parameters
% improves accuracy by minimising round off error

\section{The Landau-Lifshitz-Gilbert equation}
\label{sec:land-lifsh-gilb-normalisation}

We start from the Landau--Lifshitz--Gilbert equation with the magnetostatic, applied, exchange and magnetocrystalline (effective) fields (see Section~\ref{sec:cont-micromag}):

\begin{align}
  \pd{\Mv^*}{t^*} &= - \gymagc \Mv^* \times \Hv^* + \frac{\alpha}{M_s} \Mv^* \times \pd{\Mv^*}{t^*}, \label{eqn:llgnd} \\
  \Hv^* &= \Happ^* - \nabla^* \phi^* + \frac{2\Exchc^*}{\mu_0 M_s} \nabla^{*2} \mv + \frac{2\Kone^*}{\mu_0 M_s} (\mv \cdot \ev) \ev,
  \label{eqn:effnd} \\
  \nabla^{*2} \phi^* &= \nabla^* \cdot \Mv^*, \label{eqn:phind}
\end{align}
where we use $^*$ to denote the dimensional variables, operators and constants that will be non-dimensionalised. We assume (for simplicity) that $M_s$, $\Exchc$ and $\Kone$ are all
constant throughout the magnetic materials used.

Let
\begin{align}
  \Mv^* &= M_s \mv, \notag \\
  \phi^* &= \Phi \phi, \notag \\
  \Hv^* &= M_s \hv, \label{eqn:nddefs} \\
  \Kone^* &= \nK \kone, \notag \\
  t^* &= \frac{1}{\gymagc M_s} t, \notag \\
  x_i^* &= l x_i. \notag
\end{align}
Note that $\Mv$ and $\Hv$ have the same units so we use the same normalisation factor. For dimensional purposes derivatives are equivalent to division by the variable differentiated with respect to, so $\nabla^* = \frac{1}{l} \nabla$, $\pd{a}{t^*} = \gymagc M_s \pd{a}{t}$ etc.

Combining equation~\eqref{eqn:llgnd} with the definitions~\eqref{eqn:nddefs} gives
\begin{equation}
  \notag
  \dmdt \gymagc M_s^2 =
  - \gymagc M_s^2 \mv \times \hv + \frac{\alpha}{M_s} \gymagc M_s^3 \mv \times \dmdt,
\end{equation}

cancelling the various constants results in the non-dimensionalised Landau--Lifshitz--Gilbert equation
\begin{equation}
  \label{eq:53}
  \dmdt = - (\mv \times \hv) + \alpha (\mv \times \dmdt).
\end{equation}

Similarly for equation~\eqref{eqn:phind}
\begin{align*}
  \frac{1}{l^2} \Phi \nabla^{2} \phi &= \frac{1}{l} M_s \nabla \cdot \mv, \\
  \frac{\Phi}{M_s l} \nabla^{2} \phi &= \nabla \cdot \mv.
\end{align*}

Letting $\Phi = M_s l$ we obtain
\begin{equation}
  \label{eq:57}
  \lap \phi = \nabla \cdot \mv.
\end{equation}

Repeating the substitutions for equation~\eqref{eqn:effnd} gives
\begin{align*}
  M_s \hv &= M_s \happ - \frac{\Phi}{l} \nabla \phi + \frac{2 \Exchc }{\mu_0 M_s} \frac{1}{l^2} \lap \mv + \frac{2\kone}{\mu_0 M_s}  \nK (\mv \cdot \ev) \ev, \\
  \hv &= \frac{M_s}{M_s} \happ - \frac{M_s}{M_s} \nabla \phi + \frac{2 \Exchc}{\mu_0 M_s^2} \frac{1}{l^2} \lap \mv + \frac{2\kone}{\mu_0 M_s^2} \nK (\mv \cdot \ev) \ev.
\end{align*}

This can be further simplified by choosing the exchange length\footnote{There are actually two exchange lengths: one based on the strength of exchange as compared with the magnetostatic field and another by comparison with the magnetocrystalline anisotropy. We use the magnetostatic-field-based exchange length for normalisation to avoid division by zero in the case of zero magnetocrystalline anisotropy.} as the characteristic length scale
\begin{equation}
  \label{eqn:l-normalisation}
  l = \sqrt{ \frac{2 \Exchc}{\mu_0 M_s^2} },
\end{equation}

and
\begin{equation}
  \label{k-normalisation}
  \nK = \frac{ \mu_0 M_S^2}{2}.
\end{equation}

So the final system of equations is
\begin{equation}
  \begin{aligned}
    \dmdt &= - \mv \times \hv + \alpha \mv \times \dmdt, \\
    \hv &= \happ - \nabla \phi + \lap \mv + \kone (\mv \cdot \ev) \ev.
    \label{eqn:nd-llg-full}
  \end{aligned}
\end{equation}

\section{The Landau-Lifshitz form of the LLG}
\label{sec:land-lifsh-normalisation}

The dimensional Landau-Lifshitz equation is given in equation~\eqref{eq:LLG}:
\begin{equation}
  \label{eq:LLG-dim}
  (1 + \dampc^2) \pd{\Mv^*}{t^*} = - \gymagc \Mv^* \times \Hv^*
  - \frac{\gymagc \dampc}{M_s} \Mv^* \times (\Mv^* \times \Hv^*).
\end{equation}
The non-dimensionalisation process is essentially the same as for the Gilbert form, by substituting in the normalisations in equation~\eqref{eqn:nddefs} we obtain
\begin{equation}
  (1 + \dampc^2) \dmdt = - \mv \times \hv - \dampc \mv \times (\mv \times \hv).
\end{equation}
The field equations are obviously identical to those in equation~\eqref{eqn:nd-llg-full}.


Alternatively the time variable could be normalised differently to remove the factor of $(1 + \dampc^2)$. 
This gives us
\begin{equation}
  t^* = \frac{1 + \dampc^2}{\gymagc M_s} t,
\end{equation}
and
\begin{equation}
  \dmdt = -\mxh -\dampc \mxmxh.
\end{equation}
The $\hv$ and $\phi$ non-dimensionalisations are identical to the Landau-Lifshitz-Gilbert equation case.
However this results in time scales varying between the two equations, making comparisons more difficult.

\section{Boundary conditions}
\label{sec:non-dim-boundary-conditions}

The boundary conditions on the LLG are
\begin{equation}
  \Mv^* \times \pd{\Mv^*}{\nv*} = \frac{-1}{A} \Mv^* \times \pd{w_s^*}{\Mv^*}.
\end{equation}
Using the substitutions from above this becomes
\begin{equation}
  \begin{aligned}
    \frac{M_s^2}{l} \mv \times \dmdn &= \frac{-2}{l^2 M_s^2 \mu_0} \frac{M_s W}{M_s} \mv \times \pd{w_s}{\mv}, \\
    \mv \times \dmdn &= \frac{-2 W}{l \mu_0}  \mv \times \pd{w_s}{\mv}
  \end{aligned}
\end{equation}

??ds not sure what to do next, not important since no surface anisotropy for me...



\section{Energy calculations}
\label{sec:energy-calculations}

Sometimes we may need to know the non-dimensional energy of a micromagnetic state in a form consistent with that used for the LLG equation. One example is for the computation of an effective damping constant, see section ??ds.

We repeat the process used in Section~\ref{sec:land-lifsh-gilb-normalisation} to get a set of non-dimensionalised energy equations, starting from the equations given in Section~\ref{sec:energy-magnetic-body}:

\begin{equation*}
  \Eapp^* = - \mu_0 \int_{\magd} \Mv^* \cdot \Happ^* \d \magd^*,
\end{equation*}
\begin{equation}
  \Ems^* =  \frac{-\mu_0}{2} \int_{\magd} \Mv^* \cdot \Hms^* \d \magd^*,
\end{equation}
\begin{equation*}
  \Eex^* =  \Exchc \int_{\magd} (\nabla^* \mv)^2 \d \magd^*,
\end{equation*}
\begin{equation*}
  \Eca^* =  \Kone^* \int_\magd (\mv \cdot \ev)^2 \d \magd^*.
\end{equation*}

We substitute definitions~\eqref{eqn:nddefs}, \eqref{eqn:l-normalisation}, \eqref{k-normalisation} and $E^* = \nE e = \mu_0 M_s^2 l^d \, e$ where $d$ denotes the number of spatial dimensions:
\begin{equation*}
  \eapp = -\mu_0 M_s^2 l^d \frac{1}{\nE} \int_{\magd} \mv \cdot \happ \d \magd
  = - \int_{\magd} \mv \cdot \happ \d \magd,
\end{equation*}
\begin{equation}
  \ems = \frac{-\mu_0 M_s^2 l^d}{2} \frac{1}{\nE} \int_{\magd} \mv \cdot \hms \d \magd
  = -\frac{1}{2} \int_{\magd} \mv \cdot \hms \d \magd,
\end{equation}
\begin{equation*}
  \eex =  \frac{\mu_0 M_s^2 l^2}{2} \frac{l^d}{l^2} \frac{1}{\nE} \int_{\magd} (\nabla \mv)^2 \d \magd
  = \frac{1}{2} \int_{\magd} (\nabla \mv)^2 \d \magd,
\end{equation*}
\begin{equation*}
  \eca = \frac{\mu_0 M_s^2}{2 } l^d \kone \frac{1}{\nE} \int_\magd 1 - (\mv \cdot \ev)^2 \d \magd
  = \frac{\kone}{2} \int_\magd 1 - (\mv \cdot \ev)^2 \d \magd.
\end{equation*}
Note that we have used
\begin{equation}
  \d \magd^* = \d (x^*)^d = l^d \d x^d = l^d \d \magd.
\end{equation}



%%% Local Variables:
%%% mode: latex
%%% TeX-master: "main"
%%% End:
