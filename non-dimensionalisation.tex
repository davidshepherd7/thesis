
\section{Non-dimensionalisation}
\label{sec:normalisations-appendix}

%??ds talk about why?

% no unecessary (for the maths) parameters
% improves accuracy by minimising round off error

\subsection{The Landau-Lifshitz-Gilbert equation}

We start from the Landau--Lifshitz--Gilbert equation with the magnetostatic, applied, exchange and magnetocrystalline (effective) fields (see Section~\ref{sec:cont-micromag}):

\begin{align}
  \pd{\Mv^*}{t^*} &= - \gymagc \Mv^* \times \Hv^* + \frac{\alpha}{M_s} \Mv^* \times \pd{\Mv^*}{t^*}, \label{eqn:llgnd} \\
  \Hv^* &= \Happ^* - \nabla^* \phi^* + J \nabla^{*2} \mv + K (\mv \cdot \ev) \ev.
  \label{eqn:effnd} \\
  \nabla^{*2} \phi^* &= \nabla^* \cdot \Mv^* \label{eqn:phind}
\end{align}
where we use $^*$ to denote dimensional variables and operators.

Let the conversion between dimensional and non-dimensional variables be given by
\begin{align}
  \Mv^* &= M_s \mv, \notag \\
  \phi^* &= \Phi \phi, \notag \\
  \Hv^* &= \nH \hv,
  \label{eqn:nddefs}
\end{align}
where $\mv$ is a unit vector .

Also let the conversion factors for time and each space direction be
\begin{align}
  t^* &= \frac{1}{\gymagc \nH} t, \notag \\
  x_i^* &= l x_i \quad (\text{so } \nabla^* = \frac{1}{l} \nabla \text{ etc.}).
  \label{eqn:nddimsdefs}
\end{align}

Combining equation~\eqref{eqn:llgnd} with the definitions~\eqref{eqn:nddefs} and~\eqref{eqn:nddimsdefs} gives
\begin{equation}
  \notag
  \dmdt \gymagc M_s \nH =
  - \gymagc M_s \nH (\mv \times \hv) + \frac{\alpha}{M_s} \gymagc M_s^2 \nH (\mv \times \dmdt),
\end{equation}

cancelling the various constants results in the non-dimensionalised Landau--Lifshitz--Gilbert equation
\begin{equation}
  \label{eq:53}
  \dmdt = - (\mv \times \hv) + \alpha (\mv \times \dmdt).
\end{equation}

Similarly for equation~\eqref{eqn:phind}
\begin{align}
  \frac{1}{l^2} \Phi \nabla^{2} \phi &= \frac{1}{l} M_s \nabla \cdot \mv, \notag \\
  \frac{\Phi}{M_s l} \nabla^{2} \phi &= \nabla \cdot \mv.
\end{align}

Letting $\Phi = M_s l$ we obtain
\begin{equation}
  \label{eq:57}
  \nabla^2 \phi = \nabla \cdot \mv.
\end{equation}

Repeating the substitution again for equation~\eqref{eqn:effnd} gives
\begin{align}
  \nH \hv &= \Happ - \frac{\Phi}{l} \nabla \phi + \frac{J}{l^2} \nabla^2 \mv + K (\mv \cdot \ev) \ev, \notag \\
  \hv &= \happ - \frac{M_s}{\nH} \nabla \phi + \frac{J}{l^2 \nH} \nabla^2 \mv + \frac{K}{\nH} (\mv \cdot \ev) \ev,
\end{align}
where $\happ = \frac{\Happ}{\nH}$.

The parameters for length, $l$, and field $\nH$ are still free to be chosen. Useful possibilities for the length are the exchange length or a unit on the length scale of a dimension of the magnetic material being modelled (typically $\sim1-100$nm). For models which involve only a single magnetic material $\nH$ can be set to {\tt max}$(\Happ,\, M_s,\, \frac{J}{l^2}, \, K)$ for simplicity.

For models with more than one type of magnetic material $\nH$ must be set to some consistent value across all materials. Similarly the non-dimensionalisation of $\Mv$ must be consistent so $M_s$ probably cannot be as the conversion factor for $\Mv$ and $\mv$ will no long be a unit vector.

\subsection{Extension to the hybrid method}
\label{sec:extens-hybr-met}

\subsection{Energy equations}

We repeat the process to get a set of non-dimensionalised energy equations, starting from the equations given in Section~\ref{sec:cont-micromag}):

\begin{equation}
  F_\text{ex}^* =  \frac{\Exchc}{M_s^{*2}} \int_{\magd} (\nabla^* M_x^*)^2  + (\nabla^* M_y^*)^2  + (\nabla^* M_z^*)^2 \d \magd^*,
\end{equation}
\begin{equation}
  F_\text{ms}^* =  \frac{-\mu_0}{2} \int_{\magd} \Mv^* \cdot \Hms^* \d \magd^*,
\end{equation}
\begin{equation}
  F_\text{ap}^* = - \mu_0 \int_{\magd} \Mv^* \cdot \Happ^* \d \magd^*,
\end{equation}
\begin{equation}
  F_\text{ca}^* = \int_\magd K_u \Big( 1 - \frac{(\Mv^* \cdot \ev)^2}{\mu_0^2 M_s^2} \Big) \d \magd^*.
\end{equation}

We let $F_\text{...}^* = \nF f_\text{...}$, then after subsitution of definitions~\eqref{eqn:nddefs} and~\eqref{eqn:nddimsdefs} we have
\begin{equation}
  f_\text{ex} =  \frac{\Exchc l^{d-2}}{\nF} \int_{\magd} (\nabla m_x)^2  + (\nabla m_y)^2  + (\nabla m_z)^2 \d \magd,
\end{equation}
\begin{equation}
  f_\text{ms} = \frac{-\mu_0 M_s \nH l^d}{2\nF} \int_{\magd} \mv \cdot \hms \d \magd,
\end{equation}
\begin{equation}
  f_\text{ap} = \frac{- \mu_0 M_s \nH l^d}{\nF} \int_{\magd} \mv \cdot \happ \d \magd,
\end{equation}
\begin{equation}
  f_\text{ca} = \frac{l^d K_u}{\nF} \int_\magd \Big( 1 - \frac{(\mv \cdot \ev)^2}{\mu_0^2} \Big) \d \magd.
\end{equation}

where $d$ denotes the number of spatial dimensions and we have used 
\begin{equation} 
  \d \magd^* = \d x^{*d} = l^d \d x^d = l^d \d \magd.
\end{equation}


%%% Local Variables:
%%% mode: latex
%%% TeX-master: "main"
%%% End:
