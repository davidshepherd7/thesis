
\section{Non-dimensionalisation}
\label{sec:normalisations-appendix}

%??ds talk about why?

% no unecessary (for the maths) parameters
% improves accuracy by minimising round off error

\subsection{The Landau-Lifshitz-Gilbert equation}

We start from the Landau--Lifshitz--Gilbert equation with the magnetostatic, applied, exchange and magnetocrystalline (effective) fields (see Section~\ref{sec:cont-micromag}):

\begin{align}
  \pd{\Mv^*}{t^*} &= - \gymagc \Mv^* \times \Hv^* + \frac{\alpha}{M_s} \Mv^* \times \pd{\Mv^*}{t^*}, \label{eqn:llgnd} \\
  \Hv^* &= \Happ^* - \nabla^* \phi^* + \Exchc^* \nabla^{*2} \mv + \Kone^* (\mv \cdot \ev) \ev.
  \label{eqn:effnd} \\
  \nabla^{*2} \phi^* &= \nabla^* \cdot \Mv^* \label{eqn:phind}
\end{align}
where we use $^*$ to denote the dimensional variables, operators and constants that will be non-dimensionalised.

%% Let the unit conversions be given by
%% \begin{align}
%%  \text{space :}& 1 \, \text{m} \rightarrow l, \notag \\ 
%%  \text{magnetisation/magnetic field strength:}& 1 \, \text{A} \text{m}^{2-d} \rightarrow M', \notag \\ 
%%  \text{time :}&1 \, \text{s} \rightarrow (\gymagc M_)^{-1}, \notag \\
%%  \text{energy :}&1 \, \text{J} \rightarrow \mu_0 M_s^2 l^d.
%%  \label{eqn:unitdefs}
%% \end{align}
%% where $d$ is the spatial dimension. The factors of $l^{2-d}$ are needed because magnetisation is a \emph{density} of magnetic moments and so the units vary with dimension.

%% Note that the quantities on the right hand side are given in the same units
%% as the unit they are rescaling (this is automatic if everything is in S.I.,
%% possibly in other unit systems as well).

Let
\begin{align}
  \Mv^* &= M_s \mv, \notag \\
  \phi^* &= \Phi \phi, \notag \\
  \Hv^* &= M_s \hv, \label{eqn:nddefs} \\
  t^* &= \frac{1}{\gymagc M_s} t, \notag \\
  x_i^* &= l x_i. \notag
\end{align}
Note that $\Mv$ and $\Hv$ have the same units so we use the same normalisation factor. Also for dimensional purposes derivatives are equivalent to division by the variable differentiated with respect to, so $\nabla^* = \frac{1}{l} \nabla$, $\pd{a}{t^*} = \gymagc M_s \pd{a}{t}$ etc.

Combining equation~\eqref{eqn:llgnd} with the definitions~\eqref{eqn:nddefs} gives
\begin{equation}
  \notag
  \dmdt \gymagc M_s^2 =
  - \gymagc M_s^2 (\mv \times \hv) + \frac{\alpha}{M_s} \gymagc M_s^3 (\mv \times \dmdt),
\end{equation}

cancelling the various constants results in the non-dimensionalised Landau--Lifshitz--Gilbert equation
\begin{equation}
  \label{eq:53}
  \dmdt = - (\mv \times \hv) + \alpha (\mv \times \dmdt).
\end{equation}

Similarly for equation~\eqref{eqn:phind}
\begin{align*}
  \frac{1}{l^2} \Phi \nabla^{2} \phi &= \frac{1}{l} M_s \nabla \cdot \mv, \\
  \frac{\Phi}{M_s l} \nabla^{2} \phi &= \nabla \cdot \mv. 
\end{align*}

Letting $\Phi = M_s l$ we obtain
\begin{equation}
  \label{eq:57}
  \nabla^2 \phi = \nabla \cdot \mv.
\end{equation}

Repeating the substitution again for equation~\eqref{eqn:effnd} gives
\begin{align*}
  M_s \hv &= \Happ - \frac{\Phi}{l} \nabla \phi + \frac{\Exchc^*}{l^2} \nabla^2 \mv + \Kone^* (\mv \cdot \ev) \ev, \\
  \hv &= \frac{\Happ}{M_s} - \frac{M_s}{M_s} \nabla \phi + \frac{\Exchc^*}{l^2 M_s} \nabla^2 \mv + \frac{\Kone^*}{M_s} (\mv \cdot \ev) \ev.
\end{align*}
This can be further simplified by choosing
\begin{align}
  \Exchc^* &= \exchc l^2 M_s, \notag \\
  \Kone^* &= \kone M_s, \notag \\
  \Happ &= M_s \happ,
  \label{eqn:ndfields}
\end{align}

leaving
\begin{equation}
  \hv &= \happ - \nabla \phi + \exchc \nabla^2 \mv + \kone (\mv \cdot \ev) \ev.
\end{equation}

The parameter for length, $l$ is still free to be chosen. Useful possibilities are the exchange length or a unit on the length scale of a dimension of the magnetic material being modelled (typically $\sim1-100$nm).

Note that for models with more than one type of magnetic material then $M_s$ cannot be used as the normalising factor for magnetisation because it is non-constant. The result of this is just a factor of $\frac{M'}{M_s(\xv)}$ (where $M'$ is a the normalising factor used) in front of the damping term of the LLG equation.

%% \subsection{Extension to the hybrid method}
%% \label{sec:extens-hybr-met}

\subsection{Energy equations}

??ds I gave up on this because I didn't need it, might be wrong...


We repeat the process to get a set of non-dimensionalised energy equations, starting from the equations given in Section~\ref{sec:cont-micromag}):

\begin{equation}
  F_\text{ex}^* =  \frac{\Exchc^*}{M_s^2} \int_{\magd} (\nabla^* M_x^*)^2  + (\nabla^* M_y^*)^2  + (\nabla^* M_z^*)^2 \d \magd^*,
\end{equation}
\begin{equation}
  F_\text{ms}^* =  \frac{-\mu_0}{2} \int_{\magd} \Mv^* \cdot \Hms^* \d \magd^*,
\end{equation}
\begin{equation}
  F_\text{ap}^* = - \mu_0 \int_{\magd} \Mv^* \cdot \Happ^* \d \magd^*,
\end{equation}
\begin{equation}
  F_\text{ca}^* = \int_\magd \Kone^* (\mv \cdot \ev) \d \magd^*.
\end{equation}

%% From the rescaling of energy given in equation~\eqref{eqn:unitdefs} we have
%% \begin{equation}
%%   F_\text{...}^* = \mu_0 M_s^2 l^3 f_\text{...}
%%   \label{eqn:energynd}
%% \end{equation}

After subsitution of definitions~\eqref{eqn:nddefs} and using $F^* = \Fnorm f$

%??ds does this work in lower dimensions?

\begin{equation}
  f_\text{ex} =  \frac{\exchc l^2 M_s l^d}{l^2\Fnorm} \int_{\magd} (\nabla m_x)^2  + (\nabla m_y)^2  + (\nabla m_z)^2 \d \magd,
\end{equation}
\begin{equation}
  f_\text{ms} = \frac{-\mu_0 M_s^2 l^d}{2\Fnorm} \int_{\magd} \mv \cdot \hms \d \magd,
\end{equation}
\begin{equation}
  f_\text{ap} = \frac{- \mu_0 M_s^2 l^d}{\Fnorm} \int_{\magd} \mv \cdot \happ \d \magd,
\end{equation}
\begin{equation}
  f_\text{ca} = \frac{ - \kone M_s l^d}{\Fnorm} \int_\magd (\mv \cdot \ev)^2 \d \magd.
\end{equation}

where $d$ denotes the number of spatial dimensions and we have used 
\begin{equation} 
  \d \magd^* = \d x^{*d} = l^d \d x^d = l^d \d \magd.
\end{equation}


%%% Local Variables:
%%% mode: latex
%%% TeX-master: "main"
%%% End:
