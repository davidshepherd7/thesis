\documentclass[12pt,a4paper,pdftex]{article}

\usepackage[top=1in, bottom=1in, left=1.4in, right=1.2in]{geometry}

% Paragraphs use a vertical space rather than tabbing
\setlength{\parskip}{\medskipamount}
\setlength{\parindent}{0pt}
\usepackage[hang,flushmargin]{footmisc} % same for footnotes

% Title within the first page
\usepackage{titling}   %Reduce spacing before title
\setlength{\droptitle}{-6em}   %-6em reduces to ~normal margins

\usepackage[colorlinks=true, % colour all links black (just using
linkcolor=black, % colorlinks=false results in boxes around
citecolor=black, % links)
filecolor=black,
urlcolor=black,
breaklinks]{hyperref}


\usepackage{xr}
\externaldocument{main}


% TC:ignore
\usepackage[sort&compress, % on multiple refs sort them and write as range
capitalise, % Use Section not section etc.
noabbrev, % Use Table not Tab. etc.
nameinlink % Make the name (eg Section) part of the hyperlink
]{cleveref}
% TC:endignore

% Call subsections sections
\crefname{subsection}{Section}{Sections}
\Crefname{subsection}{Section}{Sections}

% just use (...) for equations
\crefformat{equation}{#2(#1)#3}
\crefrangeformat{equation}{#3(#1)#4--#5(#2)#6}
\crefmultiformat{equation}{(#2#1#3)}{ and~(#2#1#3)}{, (#2#1#3)}{ and~(#2#1#3)}

% Except for start of sentences where we need to say "Equations"
\Crefformat{equation}{Equation~#2(#1)#3}
\Crefrangeformat{equation}{Equations~#3(#1)#4--#5(#2)#6}
\Crefmultiformat{equation}{Equations~(#2#1#3)}{ and~(#2#1#3)}{, (#2#1#3)}{ and~(#2#1#3)}

% a reference for "this X"
\newcommand{\thisref}[1]{this \lcnamecref{#1}}
\newcommand{\Thisref}[1]{This \lcnamecref{#1}}



%% Define new commands
%% ============================================================

%% General Latex commands
%% ------------------------------
% New way to define commands that allows multiple subscripts
\makeatletter
\newcommand\newsubcommand[3]{\newcommand#1{#2\sc@sub{#3}}}
\def\sc@sub#1{\def\sc@thesub{#1}\@ifnextchar_{\sc@mergesubs}{_{\sc@thesub}}}
\def\sc@mergesubs_#1{_{\sc@thesub#1}}


% define a macro to allow multiple references to be passed to \autoref
\newcommand\autorefs[1]{\@first@ref#1,@}
\def\@throw@dot#1.#2@{#1}% discard everything after the dot
\def\@set@refname#1{%    % set \@refname to autoefname+s using \getrefbykeydefault
  \edef\@tmp{\getrefbykeydefault{#1}{anchor}{}}%
  \def\@refname{\@nameuse{\expandafter\@throw@dot\@tmp.@autorefname}s}%
}
\def\@first@ref#1,#2{%
  \ifx#2@\autoref{#1}\let\@nextref\@gobble% only one ref, revert to normal \autoref
  \else%
  \@set@refname{#1}%  set \@refname to autoref name
  \@refname~\ref{#1}% add autoefname and first reference
  \let\@nextref\@next@ref% push processing to \@next@ref
  \fi%
  \@nextref#2%
}
\def\@next@ref#1,#2{%
  \ifx#2@ and~\ref{#1}\let\@nextref\@gobble% at end: print and+\ref and stop
  \else, \ref{#1}% print  ,+\ref and continue
  \fi%
  \@nextref#2%
}

\makeatother

%% % The LyX greyedout annotation environment
%% \usepackage{color}
%% \definecolor{note_fontcolor}{rgb}{0.80078125, 0.80078125, 0.80078125}
%% \newenvironment{lyxgreyedout}
%% {\textcolor{note_fontcolor}\bgroup\ignorespaces}
%% {\ignorespacesafterend\egroup}

% Latin
% ============================================================

% trailing slash is needed so that latex knows the final . is not the end
% of a sentence.

\newcommand{\ie}{\textit{i.e.}\ }
\newcommand{\cf}{\textit{c.f.}\ }
\newcommand{\eg}{\textit{e.g.}\ }
\newcommand{\etal}{\textit{et al.}\ }
\newcommand{\etc}{etc.\ }



%% General maths commands
%% ------------------------------
\newcommand{\pd}[2]{\frac{\partial #1}{\partial #2}} % partial deriv
\newcommand{\spd}[2]{\frac{\partial^2 #1}{\partial {#2}^2}} % 2nd partial deriv
\newcommand{\ddn}[1]{\pd{#1}{\nv}} % normal derivative
\newcommand{\vd}[2]{\frac{\delta #1}{\delta #2}} % variational derivative
\newcommand{\myvector}[1]{\mathbf{#1}}
\newcommand{\goesto}{\rightarrow}

\newcommand{\E}[1]{\times 10^{#1}} % powers of 10
\newcommand{\st}{\,|\,} % ``such that'' operator in sets (vertical line with spacing)

\newcommand{\ip}[2]{\left(#1,\, #2 \right)} % inner product
\newcommand{\ltip}[2]{\ip{#1}{#2}_{\ltwo}} % l2 inner product


% Norm and abs: better spacing than if done manually with |
\newcommand{\abs}[1]{\left|{#1} \right|}
\newcommand{\norm}[1]{\lVert #1 \rVert}
\newcommand{\ltnorm}[1]{\norm{#1}_{\ltwo}}

% rescaled time
\newcommand{\that}{\hat{t}}

% d for the end of integrals (e.g. dx) with correct spacing and non-italic
\renewcommand{\d}{{\; \text{d}}}
% standard integrals
\newcommand{\intd}[2][\magd]{{\int_{#1} #2 \d {#1}}}
\newcommand{\intb}[1]{\intd[\boundd]{#1}}

% Some vector functions
\newcommand{\gv}{{\myvector{g}}}
\newcommand{\fv}{{\myvector{f}}}
\newcommand{\pv}{{\myvector{p}}}


\newcommand{\ffv}[1]{{\myvector{f}\bigb{#1}}}
\newcommand{\gfv}[1]{{\myvector{g}\bigb{#1}}}


% Some vectors
\newcommand{\av}{{\myvector{a}}}
\newcommand{\bv}{{\myvector{b}}}
\newcommand{\cv}{{\myvector{c}}}
\newcommand{\sv}{{\myvector{s}}}
\newcommand{\kvec}{\myvector{k}}

\newcommand{\ev}{\myvector{\hat{e}}}
\newcommand{\nv}{\myvector{\hat{n}}}

\newcommand{\xv}{\myvector{x}}
\newcommand{\yv}{\myvector{y}}
\newcommand{\zv}{\myvector{z}}
\newcommand{\lv}{\myvector{l}}

\newcommand{\unitv}[1]{{\hat{\mathbf{#1}}}}
\newcommand{\iv}{\unitv{i}}
\newcommand{\jv}{\unitv{j}}
\newcommand{\kv}{\unitv{k}}
\newcommand{\unitz}{\unitv{z}}

% Matrices
\newcommand{\mymatrix}[1]{\mathrm{#1}}
\newcommand{\Pm}{\mymatrix{P}}
\newcommand{\Qm}{\mymatrix{Q}}
\newcommand{\Idm}{\mymatrix{I}}
\newcommand{\Am}{\mymatrix{A}}
\newcommand{\Gm}{\mymatrix{G}}
\newcommand{\Fm}{\mymatrix{F}}
\newcommand{\Mm}{\mymatrix{M}}
\newcommand{\Jm}{\mymatrix{J}}
\newcommand{\Km}{\mymatrix{K}}
\newcommand{\Jmca}{\mymatrix{J}_\text{ca}}
\newcommand{\Jmts}{\mymatrix{J}_\text{ts}}
\newcommand{\jts}{j_\text{ts}}

% Brackets
\newcommand{\bigb}[1]{{\left( #1 \right)}}
\newcommand{\bigs}[1]{{\left[ #1 \right]}}
\newcommand{\evalat}[1]{{\left|_{#1}\right.}}

% BEM
\newcommand{\bm}{{\mathbf{\Gm}}}
\newcommand{\tri}{\vartriangle}



% 3-component vector as a list of components
\newcommand{\threevec}[3]{\begin{pmatrix} #1 \\ #2 \\ #3 \end{pmatrix} }
\newcommand{\threevecdup}[1]{\threevec{#1}{#1}{#1}}


% Big O notation
\newcommand{\order}[1]{\text{O}(#1)}
\newcommand{\porder}[1]{\quad + \order{#1}} % with larger spacing

% Differential operators
\renewcommand{\div}{\nabla \cdot} % \div is normally division
\newcommand{\grad}{\nabla}
\newcommand{\curl}{\nabla \cross}
\newcommand{\lap}{\nabla^2}

\newcommand{\compdot}{\mathbin{:}}



%% Spaces, domains and geometrical labels
%% ------------------------------
% Domain labels used
\newcommand{\magd}{\Omega}
\newcommand{\boundd}{{\Gamma}}
\newcommand{\fulld}{{\real^d}}
\newcommand{\extd}{{\Omega^c}}

% Interior/exterior labels
\newcommand{\inte}{\text{int}}
\newcommand{\exte}{\text{ext}}

\newcommand{\real}{\mathbb{R}} % real numbers
\newcommand{\complex}{\mathbb{C}} % complex numbers
\newcommand{\integers}{\mathbb{Z}} % integer numbers

\newcommand{\sob}{\mathcal{H}} % Sobelov spaces
\newcommand{\ltwo}{L^2} %L2

\newcommand{\Dfs}{\mathcal{D}} % Set of functions that obey Dirichlet boundaries
\newcommand{\Neu}{{\scriptscriptstyle{\mathcal{N}}}} % Neumann


%% Magnetics
%% ------------------------------
% Define M, H, B vectors (i.e. bold)
\newcommand{\Mv}{\myvector{M}}
\newcommand{\Hv}{\myvector{H}}
\newcommand{\Bv}{\myvector{B}}


% polar coords
\newcommand{\ruv}{\myvector{\hat{r}}} % r unit vector
\newcommand{\phiv}{\myvector{\hat{\phi}}}
\newcommand{\thetav}{\myvector{\hat{\theta}}}
\newcommand{\rv}{\myvector{r}} % r vector


% Define some common types of H-field.
% if changing these beware of components of H which are not defined here.
\newsubcommand{\Heff}{\myvector{H}}{{\text{eff}}} %effective (total)
\newsubcommand{\Happ}{\myvector{H}}{{\text{ap}}} %applied
\newsubcommand{\Hms}{\myvector{H}}{\text{ms}} % magnetostatic/demag
\newsubcommand{\Hex}{\myvector{H}}{\text{ex}} % exchange
\newsubcommand{\Hca}{\myvector{H}}{\text{ca}} % crystalline ansiotropy
\newsubcommand{\Hthm}{\myvector{H}}{\text{th}} % thermal

\newcommand{\phim}{\phi} % magnetostatic potential
\newcommand{\phione}{u} % auxilary potential
\newcommand{\phitwo}{v} % other auxilary potential

% Normalised versions of the above fields (and M)
\newcommand{\mv}{\myvector{m}}
\newcommand{\hv}{\myvector{h}}
\newsubcommand{\heff}{\myvector{h}}{{\text{eff}}} %effective (total)
\newsubcommand{\happ}{\myvector{h}}{{\text{ap}}} %applied
\newsubcommand{\hms}{\myvector{h}}{\text{ms}} % magnetostatic/demag
\newsubcommand{\hex}{\myvector{h}}{\text{ex}} % exchange
\newsubcommand{\hca}{\myvector{h}}{\text{ca}} % crystalline ansiotropy
\newsubcommand{\hthm}{\myvector{h}}{\text{th}} % thermal
\newcommand{\nH}{H_{\mathbb{n}}} % A "magnitude" of H for normalisation

% Similarly for energies
\newsubcommand{\Eapp}{E}{{\text{ap}}} %applied
\newsubcommand{\Ems}{E}{\text{ms}} % magnetostatic/demag
\newsubcommand{\Eex}{E}{\text{ex}} % exchange
\newsubcommand{\Eca}{E}{\text{ca}} % crystalline ansiotropy
\newcommand{\e}{e} % total energy
\newsubcommand{\eapp}{e}{{\text{ap}}} %applied
\newsubcommand{\ems}{e}{\text{ms}} % magnetostatic/demag
\newsubcommand{\eex}{e}{\text{ex}} % exchange
\newsubcommand{\eca}{e}{\text{ca}} % crystalline ansiotropy
\newsubcommand{\ehop}{\e}{\hop} % total due to h operator fields


\newcommand{\nE}{E_{\mathbb{n}}} % A "magnitude" of energy for normalisation
\newcommand{\nA}{\mathbb{A}} % A "magnitude" of exchange const for normalisation
\newcommand{\nK}{\mathbb{K}} % A "magnitude" of anisotropy const for normalisation

% Magnetic constants
\newcommand{\Exchc}{A}
\newcommand{\Kone}{K_1}
\newcommand{\kone}{\mathcal{K}_1}
\newcommand{\dampc}{\alpha}
\newcommand{\dampeff}{\alpha_\text{eff}}
\newcommand{\gymagc}{{\abs{\gamma_{\text{\tiny{L}}}}}}


% SI magnetic units (Kronmuller2003)
\newcommand{\Mu}{{\text{Am}^{-1}}}
\newcommand{\Hu}{{\text{Am}^{-1}}}
\newcommand{\phiu}{{\text{A}}} % magnetic potentials
\newcommand{\Bu}{{\text{T}}}
\newcommand{\gymagu}{{\text{A}(\text{ms})^{-1}}}

% Define the LLG equation (in parts then all together)
\newcommand{\dMdt}{\pd{\Mv}{t}} % define dM/dt
\newcommand{\dmdt}{\pd{\mv}{t}}
\newcommand{\dMdn}{\pd{\Mv}{\nv}}
\newcommand{\dmdn}{\pd{\mv}{\nv}}

\newcommand{\MxH}{\Mv \times \Hv} % define M x H
\newcommand{\mxh}{\mv \times \hv}
\newcommand{\mxmxh}{\mv \left(\mv \times \hv \right)}
\newcommand{\MxdMdt}{\Mv \times \dMdt}
\newcommand{\mxdmdt}{\mv \times \dmdt}
\newcommand{\llg}{\dmdt = -(\mxh) + \dampc (\mxdmdt)}

%% Finite elements/numerical models
%% ------------------------------
% Define the test and shape functions
\newcommand{\tbf}{\varphi}
\newcommand{\tbfv}{\myvector{\varphi}}
\newcommand{\test}{v}
\newcommand{\testv}{\myvector{\test}}

\newcommand{\sbf}{\psi}
\newcommand{\ts}{{\mathcal{H}^1_h(\magd)}} % my test/shape fn space
\newcommand{\sk}{{\sbf_k}}
\newcommand{\tn}{{\tbf_\ndi}}

% Indices
\newcommand{\ndi}{n} % nodal index, not sure what to have it as...
\newcommand{\eli}{e} % element index
\newcommand{\tl}{l} % time-step index

% Green's functions - general form and main parts of 2/3D Green's functions for the laplacian operator
\newcommand{\Green}[1][]{G(\xv_{#1},\yv)}
\newcommand{\Gtwod}[1][]{\ln \abs{\xv_{#1} - \yv}}
\newcommand{\Gthreed}[1][]{\frac{1}{ \abs{\xv_{#1} - \yv}}}

% subscripts used
\newcommand{\ibasis}{{i}}
\newcommand{\ibasisb}{{j}}
\newcommand{\ibasisc}{{k}}

%% Time stepping
%% ------------------------------

% time step
\newcommand{\dtn}{\dtx{n}}
\newcommand{\dtx}[1]{\Delta_{#1}}

% "value step"
% Getting bold greek requires a hack because mathbf sees it as a "symbol"
% and so doesn't change it. This uses the direct TeX solution (from google!)
\newcommand{\dyn}{\dyx{n}}
\newcommand{\dyx}[1]{\mbox{\boldmath$\delta$}_{#1}}

% Denote various time steppers
\newcommand{\AB}{\text{AB}} % Adams-Bashforth 2
\newcommand{\imr}{\text{IMR}} % Implicit midpoint
\newcommand{\tr}{\text{TR}}
\newcommand{\bdf}{\text{BDF2}}
\newcommand{\FE}{\text{FE}} % Forward Euler (like)
\newcommand{\ebdf}{\text{eBDF3}}

% Local truncation errors
\newcommand{\lte}{T_n}

% Tol
\newcommand{\toltt}{\epsilon_{\dtx{}}}

% Newton's method
% ============================================================

% tol
\newcommand{\newtontol}{\epsilon_{\text{N}}}


% Operators
% ============================================================
\usepackage{mathrsfs}
\newcommand{\myop}[1]{\mathscr{#1}}
\newcommand{\hop}{\myop{H}}
\newcommand{\hmsop}{\myop{H}_{\text{ms}}}
\newcommand{\aop}{\myop{A}}
\newcommand{\bop}{\myop{B}}
\newcommand{\cop}{\myop{C}}
\newcommand{\lop}{\myop{L}}
\DeclareMathOperator{\realp}{real}

% Fractions
% ============================================================

%% Nice one line fractions
\usepackage{xfrac}
\usepackage[ugly]{nicefrac}

\newcommand{\half}{\nicefrac{1}{2}}


% Some typewritter text
% ============================================================

% Make sure it's not italic or anything as well as being tt, because it
% looks different (and stupid). \xspace makes sure there is a space
% afterwards at the correct times without having to explicitly write the
% empty argument (i.e. \toltt{}). It can be suppressed by writing the empty
% argument.
% \newcommand{\toltt}{{\textrm{\textup{\texttt{tol}}}\xspace}}
%% Didn't want this anyway in the end


% Stuff needed for galerkin
% ============================================================
\newcommand{\skewop}{\text{\Large{$\Lambda$}}}
\newcommand{\skewm}[1]{\skewop\left[ #1 \right]}
\newcommand{\crossop}[2]{\skewm{ #1 } \cdot \left( #2 \right)}

\newcommand*\circled[1]{\tikz[baseline=(char.base)]{
    \node[shape=circle,draw,inner sep=1pt] (char) {#1};}}
\newcommand{\mxex}{I}
\newcommand{\mxmxex}{J}

\newcommand{\intp}[1]{\sum_\ibasisc \sk #1_\ibasisc}
\newcommand{\intpb}[1]{\left( \intp{#1} \right)}
\newsubcommand{\hs}{\hv}{\text{s}}

\newcommand{\ik}{\ibasisc}

% residuals
\newsubcommand{\rex}{\myvector{r}}{\text{ex}}
\newsubcommand{\rexh}{\myvector{r}}{\text{ex,h}}
\newsubcommand{\rll}{\myvector{r}}{\text{ll}}
\newcommand{\rllg}{\rv}
\newcommand{\rphi}{s}



% Notation for midpoint method stuff
% ============================================================

% t at midpoint
\newcommand{\thfx}[1]{t_{#1+\half}}
\newcommand{\thf}{\thfx{n}}

% exact y of t at midpoint
\newcommand{\yvhfx}[2]{\yv#1(\thfx{#2})}
\newcommand{\yvhf}[1][]{\yvhfx{#1}{n}}

% midpoint approximation to y
\newcommand{\yvmx}[1]{\yv_{#1+\half}}
\newcommand{\yvm}{\yvmx{n}}


% df/dy matrix
\newcommand{\dfdy}{F}
\newcommand{\dfdyhfx}[1]{\dfdy_{#1+\half}}
\newcommand{\dfdyhf}{\dfdyhfx{n}}

% error due to midpoint approx
\newcommand{\ymiderr}{a_n}

% full expressions for midpoint values
\newcommand{\mpm}{\frac{\mv_{n+1} + \mv_n}{2}}
\newcommand{\mpt}{\frac{t_n + t_{n+1}}{2}}
\newcommand{\mpdmdt}{\frac{\mv_{n+1} - \mv_n}{\dtn}}
\newcommand{\mphop}{\hop \left[ \mpm \right]}
\newcommand{\mphapp}{\happ \left(\mpt \right)}


% Commands from intermag paper 
% ============================================================

\newcommand{\dash}{\text{-}}
\newcommand{\dtinitial}{\dtx{\text{init}}}
\newcommand{\dtmax}{\dtx{\text{max}}}
\newcommand{\zerov}{\mathbf{0}}
\newcommand{\mvtemp}{\tilde{\mv}}


% Paper/thesis commands
% ============================================================

\ifthesis
\newcommand{\secpaper}{Section}
\else
\newcommand{\secpaper}{paper}
\fi


%%% Local Variables:
%%% mode: latex
%%% TeX-master: "main"
%%% End:


\title{Thesis Corrections}
\author{David Shepherd}

\begin{document}
\maketitle


\begin{enumerate}
\item
  \begin{quotation}
    Implement minor corrections (typos etc) as provided through
    separate list or marked up in thesis from both examiners.
  \end{quotation}
  I have implemented the minor corrections.

\item
  \begin{quotation}
    Chapter 7: Extend the list of methods tested to include one easily
    available library functions, such as scipy's VODE (both for 7 and
    8.1). Given the many libraries that are easily available, the
    practitioner will ask whether it is worth using (and implementing) the
    IMR method. Thus the direct comparison will embed the new IMR results
    into the research context. Furthermore, comparison of the methods
    coded by yourself with established library code will instill further
    trust in the correctness of your coding and the results.
  \end{quotation}
  I have compared scipy's \vode BDF2 integrator against \oomph's BDF2 integrator for an ODE problem and, as would be expected, the new results are almost identical. The results and some discussion are included in the chapter on the adaptive implicit midpoint rule (on page \pageref{fig:vode-osc-example}).

% ??ds first draft
  I have also added a note on page \pageref{corrections-bdf2-well-established-note} stating that the BDF2 integrator used is the main time integration scheme of \oomph and that the code has been well tested over the past $\sim 10$ years.

\item
  \begin{quotation}
    Introduce methods [re-normalisation to orbit of precession],
    [self-correcting LLG] and [tolerance based re-normalisation], somewhere
    in the thesis (for example section 3.4.8) as practical "state of the
    art" approaches that have been proposed in the community to ensure
    magnetisation length conservation.
  \end{quotation}
  The self-correcting LLG and a tolerance-based re-normalisation approach are now introduced in section \ref{sec:ensuring-constant-mv} (Magnetisation length conservation) on pages \pageref{sec:sc-llg} and \pageref{alt-ml-renorm-intro} respectively.

  On the subject of the ``normalisation to the orbit of precession'' method I have not been able to find any information in the literature beyond the bullet points in the presentation.
  I have asked Mike Donahue for more information but he was unable to provide any.
  As such I have not been able to introduce this method.

\item
  \begin{quotation}
    Review section 7.4 and 8, using the above re-normalisation techniques
    to provide better context for the IMR method. In more detail: repeat
    the computational examples with [the re-normalisation methods] and compare with the IMR.
    It seems sensible to keep the current data on (i) no re-normalisation
    and (ii) re-normalisation after every step to provide a comprehensive
    overview of the different possible approaches.
  \end{quotation}
  I have carried out detailed experiments comparing the tolerance based re-normalisation and self-correcting LLG methods with the basic method of re-normalisation after every step for the example ODE LLG problem.
  These experiments are documented in a new 9 page appendix (appendix \ref{sec:magn-renorm-meth} on page \pageref{sec:magn-renorm-meth}).
  The conclusion is that the basic re-normalisation method is clearly superior to the alternative methods, and there is no reason to expect that these results would change for other problems.
  As such there does not seem to be any reason to experiment with the alternative methods for other example problems.

% ??ds first draft
\item
  \begin{quotation}
    Consider a review of the choice of figures in chapter 8 to convey
    the intended message(s) as clearly as possible; i.e. choose figures
    that explain the value of the IMR method. Measures such as CPU time
    used overall (or Wall clock time) and memory usage are particularly
    attractive as these are key factors in day-to-day application of
    micromagnetic modelling, in addition to measures to summarise the
    accuracy of the methods. Numbers of required iterations etc are
    interesting to better understand where a benefit/disadvantage comes
    from.
  \end{quotation}
  Since the value of the IMR method is likely to come from a reduction in the global (temporal) error norm I have added additional passages to chapters 7 and 8 (on pages \pageref{cpu-correction-0}, \pageref{cpu-correction-1}, \pageref{cpu-correction-2}) detailing how differences in this error norm affect the CPU time via the ability to take larger time steps.

It would be difficult and time consuming to create figures showing this reduction in CPU time directly: in order to fairly compare the CPU time used by a time integration scheme one would have to somehow ensure that the global error norms for the results were almost identical.
Essentially this could only be done by trial and error, and may be impossible due to the non-linearity of the problem.

  Plots of the numbers of iterations are already shown for the example problems where novel solvers are used.
However, I have added descriptions of how the number of Newton-Raphson iterations and linear solver iterations affect the memory (via reduced maximum Krylov subspace size) and CPU time on page \pageref{mem-cpu-solver-correction}.

\item
  \begin{quotation}
    In the mumag standard problem 4 study, we suggest not to compare
    against d'Aquino's data as this is based on finite difference
    calculations and very likely to introduce different dynamical behaviour
    due to the different computation of demag and the other fields.
    Instead, repeat the simulation with the different integrator
    techniques, i.e IMR, and [the three alternative re-normalisation
    methods], where [each re-normalisation method] can each be combined with
    TR, BDF2 and - if desired - the CVODE approach. This will allow you to
    look at the changes in great detail and any difference is guaranteed to
    originate from the integration routine (as you use the same
    computational framework for everything else).
  \end{quotation}
  I have added additional results for the \mumag problem \#4 solved with alternative time integration methods as requested.
  A comparison of TR, BDF2 and IMR in terms of the magnetisation values and the time steps selected has been added on page \pageref{more-mumag4-correction-1}.
  On page \pageref{more-mumag4-correction-2} I have added some results comparing the accuracy of the nodal and Gaussian quadrature schemes (this is relevant to the ``comparison of different integrator techniques'' because geometric integration requires both IMR and a nodal quadrature scheme).

  However, while I agree that the comparison with finite difference methods is not ideal, I still believe that it is important to verify the model by comparing against existing results from other models for some standard problem.
  Since the \mumag problem \#4 is the only such problem for dynamic micromagnetics, and since all available data for the solution are computed using finite difference methods there is no other choice for this comparison.
  As such I would strongly prefer to leave the comparison in my thesis.

\item
  \begin{quotation}
    Rewrite the discussion of the energy conservation error for the
    triangular elements to draw out the positives. Although the
    conservation does not appear to be as good as quad elements, it is
    still better than the current state-of-the-art using standard time
    integration methods.
  \end{quotation}
  I have rewritten the discussion of energy conservation with triangular elements in a more positive manner as requested (see pages \pageref{sec:triangular-meshes}, \pageref{sec:numer-experiments-conclusions}, \pageref{sec:future-work}).

\end{enumerate}

% Were the plots in chapter 6 included as a correction or just a suggestion for the future? They don't appear in either of the typed lists of corrections and there is only a question on Andrew's handwritten copy?
% The possible correction was to replace plots by means + stddev. When you
% do this remember to also change the plots in ch. 8

\end{document}


%%% Local Variables:
%%% mode: latex
%%% TeX-master: ""
%%% End:
