


\section{An adaptive midpoint method time integrator}

%** intro

%*** other names?

%*** Banas' work

%** midpoint vs bdf2 vs trapazoid rule in micromagnetics

%** Adaptive scheme types, total error, spatial errors, LTE

%** The maths (standard rule, definitions)

Let $\yv(t)$ be a vector function of time, let $\yv_n$ denote a vector of estimates to $\yv(t)$ with $t = t_n$.
Let $\dt{n} = t_{n+1} - t_n$ be the time step.
Then given a system of equations of the form
\begin{equation}
  \pd{\yv(t)}{t} = \fv(t, \yv(t)),
\end{equation}
the midpoint method is given by
\begin{equation}
  \yv_{n+1} = \yv_n + \dt{n} \fv(\frac{t_n + t_{n+1}}{2}, \frac{\yv_n + \yv_{n+1}}{2}).
  \label{eq:basic-midpoint}
\end{equation}
Note that this is valid for both constant and non-constant times (\cf BDF2, AB2 which change).

However, commonly we wish to solve a system of equations where $ \pd{\yv(t)}{t}$ is only given implicitly. For example the Gilbert form of the Landau-Lifshitz-Gilbert equation is given by
\begin{equation}
  \llg
\end{equation}
 In this case the equivalent to \eqref{eq:basic-midpoint} is to substitute by the following
\begin{align}
  \pd{\yv(t)}{t} & \simeq \frac{(\yv_{n+1} - \yv_n)}{\dt{n}}, \\
  \yv(t) & \simeq \frac{\yv_n + \yv_{n+1}}{2}, \notag \\
  t & \simeq \frac{t_n + t_{n+1}}{2}, \notag
\end{align}
\ie the derivative is estimated using standard finite differences and everything else is evaluated at the ``midpoint''. This results in a system of equations which can be solved for $\yv_{n+1}$.

%** Derivation of an adaptive scheme

%*** How to derive LTE errors in general

%*** Details

%*** Why dfdy is ok

%** Testing

%*** Conservation properties

%*** stiffness

%*** Adaptivity effectiveness

%** Future work

%*** Does adaptivity trail off eventually like trapazoid?
