
\chapter{Introduction}
\label{sec:introduction}

An important practical application of magnetism is in the area of magnetic data storage. Hard disk drives and tapes both use the magnetisation of magnetic particles deposited on their surface to store binary encoded data. In the past 30 years the storage capacity of hard disk drives has grown even faster than the often cited Moore's law which predicts transistor density growth on semiconductor chips. \cite{McDaniel2005} However for such growth to continue new technologies will soon be needed. %% due to the trilemma of magnetic data storage, as discussed in \autoref{sec:trilemma}.
Two promising technologies are heat assisted recording and bit patterned media.

It is extremely desirable to be able to accurately model systems involving magnetic materials for both research and product design purposes. However this can be challenging because of the complex behaviour of magnetic materials. There are many sources of this complexity, for example:
\begin{itemize}
\item Many competing factors affect the behaviour of the magnetisation in different ways. Examples include: magnetostatic fields, external fields, anisotropy effects, exchange effects and temperature.

\item These factors can operate on very different time and distance scales. For example, exchange effects are very strong at very small distances but quickly disappear at larger distances, whereas magnetostatic fields can act (albeit comparatively weakly) over interstellar distances.

\item Determining the behaviour of magnetisation is almost always a fully three dimensional problem (unlike electrostatics which can often be reduced to a one-dimensional problem).
\end{itemize}

The use of fundamental physical models (such as density functional theory) to predict the behaviour of magnetic systems is difficult due to the complexity of modelling quantum mechanical effects. As such a phenomenological theory known as micromagnetics is commonly used to model magnetic materials.\cite{Coey2010} \cite{Kronmuller2003} In micromagnetics a continuum approximation is used -- we assume that everything can be modelled using continuous functions of space and time (the contribution due to individual atoms is averaged). Additionally a number of effects causing energy loss in the system are lumped together into a single damping term, simplifying the model greatly. This, however requires an empirically determined damping constant.

Micromagnetic models are extremely useful for investigations into the behaviour of magnetic systems as evidenced by the large number of citations for micromagnetics packages such as NIST's \texttt{OOMMF}\cite{oommf-website}. Additionally the list of customers using SuessCo's \texttt{FEMME} package\cite{suessco-website} contains (along with other companies) all major hard disk drive manufacturers, indicating that micromagnetic models are heavily used in the development of hard disk drives. However currently existing models are inadequate for the modelling of heat assisted recording and are often not optimally efficient for the modelling of bit patterned media.

%% In this report we first give a summary of the ``trilemma'' problem currently facing magnetic recording, and how it was last overcome by transitioning from longitudinal to perpendicular recording in \autoref{sec:trilemma} and \ref{sec:long-perp-record}. We next give an overview of micromagnetic models, methods previously used and the rationale behind the specific methods chosen for this project (\autoref{sec:cont-micromag} and~\ref{sec:numer-meth-micr}). We then give an introduction to two commonly used methods of spatial discretisation -- the finite difference and finite element methods in \autoref{sec:intr-finite-ele-diff}). Sections~\ref{sec:galerk-meth-llg} and \ref{sec:hybr-finit-elem} give a detailed description of the application of the discretisation methods to the continuous mathematical model. \autoref{sec:impl-test} gives a short discussion of the implementation details and some basic tests demonstrating that the model works as expected. Finally \autoref{sec:current-progress} and \ref{sec:objectives} discuss the progress made so far and the future objectives of my project.

%%% Local Variables:
%%% mode: latex
%%% TeX-master: "main"
%%% End:





