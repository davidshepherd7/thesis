
\chapter{Introduction}
\label{sec:introduction}

??ds lots more waffle about this stuff
There are many technologically important applications of magnetic materials, particularly in the area of data storage in hard disk drives.
There are also some promising areas for future technologies such as microwave oscillators and RAM in which the data is stable over long periods of time.

It is extremely desirable to be able to accurately model systems involving magnetic materials for both research and product design purposes.
The use of fundamental physical models (such as density functional theory) to predict the behaviour of magnetic systems is extremely computationally expensive due to the difficulty of modelling quantum mechanical effects on the scales required.
As such theory known as micromagnetics is widely used to model magnetic materials \cite{Coey2010} \cite{Kronmuller2003}.
In micromagnetics a continuum approximation is used -- we assume that everything can be modelled using continuous functions of space and time (\ie the contribution due to individual atoms is averaged out).
Also a semi-classical approximation is used: effects which are quantum mechanical in origin, such as the exchange interaction, are approximated using classical physics.
Finally a number of different effects causing energy loss in the system are modelled by a single damping term with an empirically determined strength.

Micromagnetic models are extremely useful for investigations into the behaviour of magnetic systems as evidenced by the large number of citations for micromagnetics packages such as NIST's \texttt{OOMMF} \cite{oommf-website}.
Additionally the list of customers using SuessCo's \texttt{FEMME} package \cite{suessco-website} contains (along with other companies) all major hard disk drive manufacturers, indicating that micromagnetic models are heavily used in the development of hard disk drives.

Micromagnetic models can be broadly split into two categories: energy based models and dynamic models.
Energy based models aim to find stable minimum energy states for the system whereas
dynamic models simulate the evolution of the magnetisation over time.
Dynamic models require the solution of a differential equation, known as the Landau-Lifshitz-Gilbert equation (LLG).
??ds mention stochastic-ness here?

??ds mention limitations of current models? -- are there any limitations?...

\section{Aims}

In this thesis we study numerical methods with the final goal of finding more reliable and efficient methods for dynamic micromagnetics simulations.
In particular we focus on methods which improve the efficiency of so-called Geometric integration methods.
Geometric integration methods are able to retain important qualitative properties of a differential equation in the approximate solution generated using numerical techniques.
In other areas such methods have been shown to greatly reduce the overall build-up of numerical errors \cite[77]{Iserles2009}.
This allows either more accurate results at the same computational cost, or the use of coarser approximations (reducing the computational cost) without loss of accuracy.

In particular we focus mainly on a widely known time integration scheme with geometrical integration properties when applied to dynamic micromagnetics calculations: the implicit midpoint rule.
We also focus on methods in which spatial discretisation is handled using the finite element method, such methods are well suited to the study of nano-structured materials.



\section{Contents of thesis}

The first two chapters \cref{sec:cont-micromag,sec:numer-meth-micr} comprise a basic introduction to micromagnetic models and the numerical methods commonly applied.
In particular \cref{sec:time-discretisation} contains a detailed description of the properties of a selection of time integration schemes.
\Cref{sec:galerk-meth-llg} gives a more detailed introduction to the finite element method, including a description of how it can be applied to dynamic micromagnetics simulations.
The chapter ends with a discussion of how the basic finite element method can be extended to retain the geometric integration properties of them implicit midpoint rule.
\Cref{sec:hybr-finit-elem} introduces the hybrid FEM/BEM method, a widely used technique for the accurate calculation of magnetostatic fields.

The later chapters contain the research contribution of this thesis.
\Cref{sec:solution-strategies} describes efficient techniques for the solution of the coupled systems resulting from the use of FEM/BEM magnetostatics calculations with a dynamic micromagnetic problem.
In particular we introduce techniques which reduce the development of an efficient solver for a monolithically coupled model to the development of an efficient solver for the LLG alone.
Such a coupling strategy is required to retain the geometric integration properties of the implicit midpoint rule.

In \cref{sec:adaptive-imr} we introduce a novel adaptive algorithm for the implicit midpoint rule which is not specific to the LLG; to our knowledge this is the first such algorithm.
The same chapter also contains numerous numerical experiments demonstrating its effectiveness on a number of ordinary differential equation problems and the geometric integration properties when applied to the LLG.

In the final two chapters we present a number of numerical experiments using the entirety of the models developed in this thesis.
In \cref{cha:numer-experiments} the model is validated against a number of examples: a wave-like problem with an analytical solution, relaxation under a non-uniform field and the \mumag standard problem \#4.
Additionally the convergence and geometric integration properties of a variety of time integration schemes are compared for these problems.
In \cref{cha:stiffn-llg-equat} the comparative efficiency of two classes of time integration schemes (implicit and explicit) are compared for an example problem across a range of spatial discretisations.

\Cref{cha:analyt-solut-land} contains a description of two analytical solutions to the LLG equation which have proven very useful in the verification of our implementation.
The other appendices contain technical details of some derivations.


%%% Local Variables:
%%% mode: latex
%%% TeX-master: "main"
%%% End:
