\chapter{Numerical Methods for Dynamic Micromagnetic Modelling}
\label{sec:numer-meth-micr}

In this Chapter we give an overview of numerical methods that have or could be used in dynamic micromagnetic calculations.
These calculations involve solving some form of the LLG (or sometimes a related equation to incorporate additional physics) with fields determined by energy derivatives as discussed in \autoref{sec:cont-micromag}.
These systems of partial differential equations (PDEs) can only be solved analytically in a few extremely simple cases,\footnote{For example the Stoner-Wolfarth theory for the rotation of a single grain.\cite{Stoner1948a}}\cite{Aharoni1996} so numerical solution methods are almost always required.

Many numerical methods for solving non-linear PDEs, such as \eqref{eq:LLG}, can be thought of as a combination of various component methods,\footnote{Technically proofs of convergence etc. should be carried out for every combination of time and space discretisation\cite[382]{Iserles2009}; in practice suprises seem to be rare, at least within micromagnetics.} each of which handles a different part of the conversion from a purely mathematical description into a complete algorithm which can be performed by a computer.

The essential components of such a numerical method are:
\begin{itemize}
\item Spatial discretisation: convert space derivatives into algebraic relationships between points in space, \eg Finite elements, finite differences, macrospins. 
\item Time discretisation (a.k.a. time integration):  convert time derivatives into algebraic relationships between points in time, \eg Runge-Kutta methods, backwards difference formulae (BDF), implicit midpoint rule (IMR).
\end{itemize}
For some choices of time and space discretisations additional components are needed:
\begin{itemize}
\item Linearisation: Convert a system of non-linear algebraic equations into a sequence of systems of linear algebraic equations, \eg Newton-Raphson method, Picard/fixed-point iteration. 
\item Linear solver: solve a system of linear algebraic equations, \eg LU decomposition, Krylov solvers, multigrid methods. 
\end{itemize}

Additionally in micromagnetic modelling the calculation of the magnetostatic field is required.
If the integral form given in \eqref{eq:Hmsint} is used then an additional integral evaluation method is needed to handle it.
However if the equivalent potential formulation \eqref{eq:Hms}-\eqref{eq:phi-inf} is used then we simply have another PDE to add to the system of PDEs describing the magnetisation dynamics.

It should be noted that not all methods fit into this framework of independent methods for each part of the problem.
In particular one example which uses a combined space/time discretisation to obtain some interesting properties is discussed (briefly) in \autoref{sec:advanced-lin}.
In general however, the modularity of both understanding and code given by thinking of a PDE solver as a combination of components is extremely powerful.

Diagrams showing how the various methods are related are given for space discretisation in \autoref{fig:types-spat-discl}, time discretisation in \autoref{fig:types-time-disc}, and magnetostatic field calculations in \autoref{fig:types-mag-stat}.


\section{Spatial Discretisation}
\label{sec:spat-discr}

\subsection{Macrospins}
\label{sec:sd-macrospins}

??ds does this really belong here? macropins kind of not micromagnetics...

In a granular material (a material consisting of magnetic grains separated by a non-magnetic material%% , see \autoref{fig:Layouts-for-magnetic}
) the simplest way to discretise the problem is to assume that within each grain the exchange coupling is so strong that it rotates as a single \emph{macrospin}. We assign a single value of $\Mv$ to each macrospin and proceed to calculate energy, effective field and/or magnetisation of each macrospin as required. One caveat is that the magnetostatic self field is not automatically accounted for since there is no modelling of intra-grain effects. Hence it must be calculated and added separately to the magnetostatic interactions between grains. When applied like this the magnetostatic self field is often called the \emph{shape anisotropy} since it is dependant on the shape of the grain and acts very similarly to the magnetocrystalline anisotropy. The same approach may be used with any system in which there are a number of ``small'',\footnote{All dimensions of the bodies must be much smaller than the exchange length so that all magnetisation within the body is approximately parallel.} separate magnetic bodies with approximately uniform magnetisation inside the body.

The obvious downside of a macrospin approach is that it only applies to fairly specific geometrical cases, although the case of a granular media has been of much interest for magnetic data storage. Additionally, if there are non-uniformities in magnetisation within the regions where it has been assumed constant the model may be inaccurate. However it is often simpler to construct a macrospin model than to use the methods described in \autoref{sec:sd-finite-diff-meth} and~\ref{sec:sd-finite-elem-meth}. Also the assumption that each grain has uniform magnetisation will reduce the number of calculations needed.

This approach was applied by, for example, J. Zhu and H. Bertram to study magnetisation dynamics in thin film granular media.\cite{Zhu1988}

Closely related to macrospin models are atomistic models\cite{Evans2014}.
In these models the assumption of a continuous magnetisation $\mv(\xv, t)$ is dropped in favor of an assumption that each atom is the location of a single macrospin.
As such they allow modelling of some systems that are inaccessible using micromagnetics such as ??ds.
However the required spatial resolution comes with a large increase in computational cost.
Due to the differences in the underlying model the numerical methods used are often quite different so we won't discuss them much.

\begin{figure}[\figpos]
  \centering
  \begin{tikzpicture}[level 1/.style={sibling distance=5.2cm},level 2/.style={sibling distance=5cm}]
    \node[block] {\textbf{Spatial Discretisation}}
    child {node[block] {Finite Difference}}
    child {node[block] {Macrospins}}
    child {node[block] {Finite Element}};
  \end{tikzpicture}
  \caption{Spatial discretisation schemes used in micromagnetic models.}
  \label{fig:types-spat-discl}
\end{figure}


\subsection{Finite Difference Methods}
\label{sec:sd-finite-diff-meth}

Another method of spatial discretisation is the finite difference method: a single magnetisation vector is assigned to each point on (or ``cell'' in) a simple square/cubic grid which covers the system being modelled. The method is described in more detail in \autoref{sec:intr-finite-ele-diff}.

The finite difference method works well for very simple geometries when the grid can be lined up with all geometric features. 
For example when we are interested in how the magnetisation evolves over time in a non-granular cuboid-shaped thin film of magnetic material a finite difference method will be sufficient (\eg in the $\mu$mag standard problems\cite{mumag-website}). 
However when the geometry involves curves, diagonals, hexagonal grains, bit patterned media or any other more complex geometric system other methods are better suited.

NIST's \texttt{OOMMF} model, one of the oldest micromagnetic models still in use, uses the finite difference method\cite{oommf-website}.


\subsection{Finite Element Methods}
\label{sec:sd-finite-elem-meth}

A more complex method of spatial discretisation is the finite element method\cite{HowardElmanDavidSilvester2006}.
Here the magnetic body is divided up into a finite number of polygonal \emph{elements}, which can vary in size and shape.
Within each of these elements each magnetisation component is assumed to be a simple, usually linear, polynomial function.
In practice the polynomials are defined by the values at the points where these elements meet, known as the \emph{nodes}.
Values at any other point in the domain can be calculated by interpolation between the nodes. More details of the method are given in \autoref{sec:intr-finite-ele-diff}.
When applied to a uniform cubeoid mesh of zero-th order polynomials the finite element method results in the same system of equations as the finite difference method\cite{??ds}.

The main advantage of the finite element method is that it can accurately approximate any geometrical feature by an appropriate arrangement of the polygonal elements.

An additional advantage is that the size of elements (and thus the accuracy of the approximation) can be varied arbitrarily as needed to give better accuracy in more complex or important regions. 
The choice of element size can be done automatically using \emph{adaptive mesh refinement}: after each calculation an error estimate is calculated (known as \emph{a posteriori} error estimation).
If the error is determined to be too high anywhere the mesh is refined near that region and the calculation is repeated.
Hence, given the desired error and a method to estimate the error, a mesh giving an efficient and accurate approximation can be automatically generated.\cite{Schrefl1999}

The major downside of finite element models is that the underlying mathematics is more complex than in finite difference methods.
Also the set up time and memory usage can be greater because of the additional ``bookkeeping'' required to keep track of the more complex meshes.

Finite element methods are used in the \texttt{magpar}\cite{Scholz2003}, \texttt{nmag}\cite{Fischbacher2007} and \texttt{FEMME} micromagnetics models\cite{suessco-website}.



\section{Time Discretisation}
\label{sec:time-discretisation}

After applying a spatial discretisation,we obtain a semi-discrete version of  equation~\eqref{eq:LLG}: it gives a continuous value in time of $\dMdt$ at fixed discrete points in space. To make it fully discrete, so that we can numerically solve for the time evolution of $\Mv$, we need to apply a time discretisation scheme. This section only relates to dynamic micromagnetics since energy based methods do not include time dependence.

The time discretisation methods discussed here all bear a strong similarity to the finite difference method discussed in \autoref{sec:finite-diff-appr}, except that the independent variable is time instead of space.

To explain the different time discretisation schemes we use a simple ordinary differential equation (an initial value problem)
\begin{equation}
  \begin{aligned}
    \frac{dy}{dt} &= f(t,y(t)) \quad t \in [0,T], \\
    y(0) &= y_0.
    \label{eq:45}
  \end{aligned}
\end{equation}

where $f(t,y)$ is a known function and $T$ is the end time. The idea is to use the known values of $y(t)$ at the current/previous times along with the derivative to approximate the value $y(t+h)$  after stepping forwards in time by $h$.

Some key attributes of a time discretisation scheme are:\cite{Atkinson2009}
\begin{itemize}

\item \textbf{Accuracy (order)} -- An estimate of how rapidly the error decreases as the time-step, $h$, is reduced.

\item \textbf{Stability} -- A scheme is stable if the approximated solution stays close to the exact solution, even after a large number of time-steps. 
A scheme is called conditionally stable if it is stable only for time-steps smaller than some maximum time-step or unconditionally stable if it is stable even for very large time-steps (although for very large time-steps the accuracy may be compromised).

% \item \textbf{Ability to deal with stiffness} -- Some ODEs have terms which vary on very different time scales, this is referred to as stiffness. Stiff ODEs causes some solvers to require extremely small time-steps in order to remain stable. The Landau--Lifshitz--Gilbert equation is sometimes stiff because the precession and damping terms usually operate on very different timescales but are both important for determination of the dynamics.\cite{Fidler2000}

\item \textbf{No spurious modes} -- Many schemes have ??ds additional roots.
This can cause incorrect (numerical) damping of oscillations in the solutions.

\item \textbf{Preservation of geometrical properties} -- Some differential equations have properties which should ideally be conserved in the discretised system. For example $\abs{\Mv}$ should remain constant over time in the Landau--Lifshitz--Gilbert equation but this property is often lost after discretisation.\cite{DAquino2005}

\item \textbf{Self-starting} -- A scheme is self starting if it only requires a single initial value. This is desirable because methods of estimating additional initial values may introduce errors. However more advanced schemes often require values at multiple times and so need multiple initial values.
\end{itemize}

\begin{figure}[\figpos]
  \centering
  \resizebox{\textwidth}{!}{
    \begin{tikzpicture}[level 1/.style={sibling distance=5cm},level 2/.style={sibling distance=4.9cm},level 3/.style={sibling distance=4cm}]
      \node[block] {\textbf{Time Discretisation}}
      child{node[block] {Explicit}
        child{node[block] {RK4}}
        child{node[block] {Adams-Bashford etc.}}
      }
      child{node[block,yshift=-3.3cm] {Implicit}
        child{node[block] {BDF[N]}}
        child{node[block] (midpm) {Midpoint Method}}
      }
      child{node[block] (geom) {Geometric}
        child{node[block,xshift=1.2cm,yshift=-1.5cm] {Caley Transform (applied to a discretisation method)}}
      };
      \draw[line] (geom) -- (midpm);
    \end{tikzpicture}
  }
  \caption{Some time discretisation methods  used in micromagnetics.}
  \label{fig:types-time-disc}
\end{figure}

\subsection{Explicit Schemes}
\label{sec:explicit-schemes}

Explicit time discretisation schemes give the value at some future time in terms of the value at the present time and/or previous times. The simplest such scheme is the (forward) Euler method
\begin{equation}
  \label{eq:44}
  y(t_{n+1}) = y(t_n) + h f(t_n,y(t_n)),
\end{equation}
where $h$ is the time-step. Clearly, given $f(t,y)$ and an initial value for $y(t_0)$ we can solve for $y(t_n)$ for any $n$. However the stability and convergence behaviour of this simple scheme is less than impressive. Typically more advanced explicit schemes are used which give increased stability.\cite{Atkinson2009} Also the so called ``CFL condition'' (Courant-Freidrich-Lewy condition) can force the use of lower time-steps in explicit solvers if a finite element/difference spatial discretisation with small elements/cells is used.

Micromagnetics solvers for non-stiff systems commonly use the RK4 (fourth order Runge-Katta) method.\cite{Suess2002}


\subsection{Implicit Schemes}
\label{sec:implicit-schemes}

Implicit time discretisation schemes allow much longer time-steps to be used without loss of stability. However an implicit scheme gives the value at the next time-step in terms of current/previous times and in terms of the \emph{value at the next time-step}. Hence at each step a (linear or non-linear) system of equations must be solved, however the increase in maximum time-step size offsets this increase in calculation time per step in many cases.

The backwards differences schemes are a simple and commonly used example. The first order BDF formula is
\begin{equation}
  \label{eq:48}
  y(t_{n+1}) = y(t_n) + hf(t_{n+1}, y(t_{n+1})).
\end{equation}

Because a system of equations must be solved at each step a \emph{preconditioner} may be needed to ensure the system can be solved in optimal time (\ie $\order{N}$). Magnetostatic field calculation by the hybrid method can cause difficulties in the solution of the system of equations because it adds a dense sub-block to the otherwise sparse system.

Micromagnetics models commonly use BDF schemes of various order\cite{Suess2002} for the modelling of stiff systems. The midpoint method is another possible choice.\cite{DAquino2005}

Other micromagnetic models use a combination of implicit and explicit schemes: everything except for the magnetostatic field is discretised as normal using an implicit scheme, the magnetostatic field is updated (using an explicit calculation) after each time step. This method gains a larger time-step from the implicit method but the system of equations to be solved remains sparse. However when using this method the determination of the time-step must be done specially (\ie not using normal adaptivity) since the system of equations contains no information on the magnetostatic field.\cite{Schrefl1997}

\subsection{Adaptivity}
\label{sec:adaptivity}

Adaptive time discretisation methods vary the time-step in response to an estimate of the local truncation error. This is especially computationally efficient when the magnitude of the time derivative varies widely over time, for example if the magnetisation direction changes slowly until some unknown time when it rapidly switches.

\subsection{The Non-Convex Constraint}
\label{sec:ensuring-constant-mv}

Note that in equations~\eqref{eq:LL}, \eqref{eq:Gilbert} and \eqref{eq:LLG} the direction of $\dMdt$ is always perpendicular to the current value of $\Mv$ (since all terms contain cross products with $\Mv$). Hence we have an implicit condition: $\abs{\Mv} = M_s$. However in the discretised approximation this is often lost and must be enforced separately. This condition is often called a  \emph{non-convex constraint}\footnote{Intuitively a convex set is one such that given two members of the set all points on a straight line between them are also members of the set. With the condition $\abs{\Mv}=M_s$ the set of possible values of $\Mv$ is the surface of a sphere with radius $M_s$ which is not a convex set, hence the name.}. This can pose difficulties in the numerical solution of the Landau--Lifshitz--Gilbert equations because the approximations used do not necessarily respect the constraint. Hence the solution can end up with $|\Mv| \neq M_s$ which is un-physical (for constant temperature models).

A simple method of dealing with the constraint is to re-normalise $\Mv$ after some number of time-steps or when the error in $|\Mv|$ exceeds some tolerance.\cite{Fidler2000} However this approach fundamentally changes the system of equations being solved.\cite{Lewis2003}

If we have a system with only a single (macro)spin (\ie a single value of $\Mv$ represents the magnetisation of the entire system) it is easy to avoid this problem by using a spherical polar coordinate system $(r,\theta,\phi)$. Equation~\eqref{eq:LLG} can be expressed in terms of only the angles $(\theta,\phi)$ representing the direction of $\Mv$ and the non-convex constraint is automatically enforced since
\begin{equation}
  \label{eq:40}
  \pd{\abs{\Mv}}{t} = \pd{r}{t} \equiv 0.
\end{equation}
However to extend this to systems where $\Mv$ varies with space we have to use a separate spherical polar coordinate system at each point where $\dMdt$ is calculated. Also a Cartesian global coordinate system is still needed to calculate the interactions between the discretised points (\ie magnetostatics, exchange coupling). Hence we have to convert back and forth between coordinate systems during the simulation.\cite{Scholz2003} Finally problems can occur with this approach as the polar angle, $\theta$, approaches zero because $\pd{\Mv}{t} \propto \frac{1}{\sin(\theta)}$.\cite{Fukushima2005}

% Some people have used special test functions to keep $\abs{\Mv}$ fixed, don't understand that method yet though

``Geometrical'' integration schemes aim to solve this problem by constructing a time discretisation scheme that naturally preserves the value of $|M_s|$. An example of such a scheme is the midpoint method as used by d'Aquino. The midpoint method also has other desirable properties -- it conserves energy when the damping term is zero and it ensures that the energy is a decreasing function of time when the damping is non-zero.\cite{DAquino2005} Alternatively geometrical integration methods based on Cayley transforms can be used.\cite{Lewis2003}\cite{Bottauscio2011}


\subsection{Specific micromagnetics time discretisation schemes}

??ds write something about the work summarised in Cimrak's review.


\section{Linearisation}
\label{sec:linearisation}

When implicit time integration methods are used (on a non-linear differential equation such as the LLG) a non-linear system of equations must be solved.
Explicit time integration methods essentially sidestep the non-linearity of the LLG equation by avoiding the need to solve any kind of system.

Such a problem can be stated as:
Given a non-linear operator $F$ and a list of previous magnetisation values $\mv_{\text{history}}$ (as needed for the time integration method) find
\begin{equation}
  \label{eq:non-lin-system}
  \mv_{n+1} = \min_{\mv'} \norm{\rv(\mv', \mv_{\text{history}})}.
\end{equation}


\subsection{Picard Iteration}
\label{sec:picard}

aka functional iteration, fixed point iteration

Cheap iterations

Convergence proof via Banach fixed point theorem for sufficiently small time steps.

Useful for non-stiff problems (convergence requirements similar to stiffness requirements) \cite{Iserles2009}.

In practice people often terminate the Picard iteration after only one or two steps, leading to a predictor-corrector method. 
These methods have similar properties to explicit methods: cheap per step but vunerable to stability problems.


\subsection{Newton-Raphson method}
\label{sec:newt-raph}

\newcommand{\resi}{\rv}
\newcommand{\jac}{\mathrm{J}}
\newcommand{\nowm}{\mv^0}
\newcommand{\nextm}{\mv}
\newcommand{\corr}{\delta}

The motivation for the Newton-Raphson method comes from a simple Taylor expansion of the residual. If the root of the residual is $\nextm$ and we have an initial guess for this root $\nowm$ then we can obtain a correction to this initial guess from:
\begin{equation}
  \begin{aligned}
    0 &= \resi(\nextm) = \resi(\nowm + \corr) \\
    &= \resi(\nowm) + \jac(\nowm) \cdot \corr + \order{\corr^2},
  \end{aligned}
\end{equation}
where $\jac = \pd{\resi}{\mv}$ is the Jacobian matrix for the residual.
So
\begin{equation}
  \label{eq:49}
  \nextm = \nowm + \corr = \nowm + \jac^{-1}(\nowm) \cdot \resi(\nowm) + \order{\corr^2}.
\end{equation}

This suggests the iteration
\begin{equation}
  \mv^{i+1} = \mv^i + \jac^{-1}(\mv^i) \cdot \resi(\mv^i),
\end{equation}
in which the error is squared after each iteration (assuming that the Taylor series expansion is valid and that we can discard the higher order terms).

In time stepping problems the value at the previous time step provides a good initial guess, $\mv^0$.
For micromagnetics problems we find that this reliably converges, to a tolerence of $\sim1\E{-10}$, in around two steps!

The downside of this method is that it requires the assembly and solution of a system of linear equations to obtain $\corr$.
This is the subject of \autoref{sec:solution-lin-sys}.


\subsection{Advanced LLG-specific methods}
\label{sec:advanced-lin}

Linearisation based on choice of test functions of much interest in recent times, initially for only exchange effective field\cite{Alouges2008}.
Has recently been extended to handle general effective field contributions\cite{Banas2012}.

Advantages:
\begin{itemize}
\item Only need to solve one linear system per step (rather than $\sim 2$ for Newton's method).
\item Jacobian matrix is constant.
\end{itemize}

Disadvandages:
\begin{itemize}
\item Method is limited to first order: so need step sizes orders of magnitude smaller.
\item Have to use specialised time integration scheme (no geometric, adaptive, ... schemes).
\item Specific to the LLG equation--standard code libraries may not work, not as widely understood as standard FEM, extension to related equations (stochastic LLG, LLGS, LLB) is non-trivial.
\end{itemize}

The limitation to first order has been worked around recently, however the higher order schemes lose either stability or linearity\cite{Kritsikis2014}.
Since the advantage of the scheme over standard methods was this unique combination of properties, this appears (to me at least) to not be very useful.

Still a promising line of research though!


\section{Solution of linear systems}
\label{sec:solution-lin-sys}

If we are using an implicit time integration scheme with the Newton-Raphson method for linearisation then we are left with the problem of solving a sequence of sparse linear systems.
The problem can be stated as: Given a sparse $n \times n $ matrix $\Am$ and a vector $\bv$ find $\xv$ such that
\begin{equation}
  \label{eq:linear-system}
  \Am \xv = \bv.
\end{equation}
This is a very well studied problem, see for example\cite{Saad2000}, but efficent techniques for very large $n$ (which corresponds to a large spatial problem and/or good spatial resolution) are complex and problem dependant.


\subsection{Direct methods}
\label{sec:direct-methods}

Construct exact solutions to linear systems of equations

Main example is LU decomposition.

The main advantage of direct methods is their robustness: given any non-singular linear system a well designed direct solver will be able provide an answer.
As such direct solvers are usually the first approach tried when solving a new system of equations.

However as matrix sizes become large direct solvers become increasingly demanding in both time and memory.
The problem is that the inverse of a sparse matrix is in general dense, or at least significantly more dense than the starting matrix.
This means that the time and memory requirements for the solution scale as $\order{n^2}$, where $n$ is the number of rows/columns in the matrix.
e


\subsection{Krylov Solvers}
\label{sec:krylov-solvers}

Begin with an initial guess and move closer to the solution by iteration, until error is reduced by some amount.

Based on Krylov subspace (polynomial of $\Am$ multiplied by $\xv$).

Can be extremely efficent

Rely on effective preconditioning, can also be thought of as convergence acceleration for another method--the preconditioner.


% \subsection{Preconditioning}

% \begin{itemize}
% \item If implicit time-stepping is used solution of the linear systems created at each time-step can be troublesome.
% \item We can speed this up using a preconditioner.
% \item Some work has been done on preconditioning by Suess et. al. \cite{Suess2002}.
% \item Also Banas et. al.\cite{Banas2008} \cite{Banas2010} used a multigrid preconditioner for the Maxwell-LLG equation( magnetostatic field is computed via Maxwell's equations, exchange field is also accounted for, crystalline anisotropy is not). The use of Maxwell's equations introduces complications because of the curls.
% \end{itemize}


\section{Magnetostatic Field Calculations}

The naive method of evaluation results in a double integral (or a sum after spatial discretisation) over all of the magnetic body. 

This is usually unreasonably slow so more advanced methods are needed. Such methods break down into two categories: methods based on quickly approximating the integrals and methods based on the use of a \emph{scalar potential} to convert the calculation into a form similar to the rest of the problem.

\subsection{Integral Methods}
\label{sec:magstat-field-calc-inte}

\begin{figure}[\figpos]
  \centering
  \begin{tikzpicture}[level 1/.style={sibling distance=5.4cm},
    level 2/.style={sibling distance=3.6cm}]

    \node[block] {\textbf{Magnetostatic Calculations}}
    child {node[block,text width=6cm] {Scalar Potential Formulation (with some spatial discretisation)}
      child{node[block,text width=4cm,xshift=-1cm] {Asymptotic Boundary Conditions}}
      child{node[block,text width=4.3cm] {Hybrid Finite/Boundary Element Method}}
    }
    child {node[block,yshift=-2.7cm] {Integral Formulation}
      child{node[block,text width=3.2cm] {Full Calculation}}
      child{node[block,text width=3.2cm] {Fast Fourier Transform}}
      child{node[block,text width=3.2cm] {Fast\\ Multipole\\ Method}}
    };
    \end{tikzpicture}
  \caption{Methods of magnetostatic field calculation that have been used in micromagnetic models.}
  \label{fig:types-mag-stat}
\end{figure}

After the application of a discretisation scheme the integrals in equation~\eqref{eq:Hmsint} become a sum over all nodes. The naive way to calculate the magnetostatic fields would then be to work through the list of nodes calculating the field at each of them. Then for each node a contribution from all other nodes needs to be calculated. Hence this results in an algorithm complexity that scales as $\order{N^2}$ (where $N$ is the number of points used in the space discretisation) which is usually unacceptably slow.

These problems are even worse in the case of implicit time integration methods: the integral formulation couples mgnetisation at every point to every other point, this means that the Jacobian representing the derivatives of each magnetisation with respect to each other magnetisation is completely dense (and hence solution times are extremely slow)!
One way around this is to discard the magnetostatic interactions from the Jacobian\cite{DAquino2005}, but in this case we no longer have a real Newton method.
Because of this convergence is significantly slower and may even fail for large systems.
Explicit time integration methods avoid these issues because no system solve is required.

\subsubsection{Fast Fourier Transform Methods}

If the individual magnetic charges are on a regular lattice (or approximated by a regular lattice) and the boundary conditions are periodic, then the redundancy can be exploited to speed up the magnetostatic field calculations. The calculation of $\Hms$ in equation~\eqref{eq:Hmsint} can be thought of as applying a convolution operator $D$ (\ie $\Hms = D \big[\Mv\big]$). The matrix corresponding to this operator  is only dependant on geometry, hence it can be precomputed, Fourier transformed and stored for use in the main simulation. Then all that is needed to calculate the magnetostatic field is to apply a Fourier transform to $\Mv$, compute the convolution and transform the result back into the time domain by applying the inverse Fourier transform. Because of the regularity, applying the convolution in the frequency domain is very fast and hence the complexity of the calculation is limited by the complexity of a fast Fourier transform, which is $\order{N \log(N)}$.\cite{Jones1997}

The downside of this method is that points to be calculated must be on a regular lattice, similar to the finite difference method. Hence, it is most suited for use in combination with models using a finite difference spatial discretisation. Alternatively it may be used in less regular macrospin models by approximating the the macrospins as being on a regular lattice.\cite{Jones1997}

A fast Fourier transform method is used to calculate the magnetostatic field in \texttt{OOMMF}.\cite{oommf-website}

\subsubsection{Fast Multipole Method}
\label{sec:fast-mult-meth}

An alternative method of calculation of the magnetostatic field is the multipole method. It takes advantage of the fact that distant magnetic charge has a much smaller effect on the total field at a point than nearby magnetic charge.

For the field calculation at a specific point, $\xv$, the full calculation is only performed for nearby magnetic charges. Groups of more distant charges are approximated (lumped) as a single multipole placed at the centre of the group. As the charges become more distant they contribute much less to the field due to the $\frac{1}{(\xv - \xv')^2}$ scaling in equation~\eqref{eq:Hmsint}. Hence for distant points the multipole approximation can become less accurate, and so faster to calculate, while still retaining the required overall level of accuracy.

The trick for quickly calculating fields at a large number of points is to pre-calculate the multipole approximations for a range of accuracies over all space. Then the calculation of a field at a single point only requires the full calculation of effects from a few nearby points and from the appropriate multipoles.\cite{Beatson}

One advantage of this method over the fast Fourier transform is that it allows for arbitrary geometries. Also the complexity of the method is $\order{N}$, where $N$ is the number of magnetic charges (equivalent to the number of nodes/cells/macrospins after spatial discretisation).\cite{Chang2011}

The fast multipole method is used, with massive parallelisation for GPUs, to quickly calculate magnetostatic fields in FastMag.\cite{Chang2011} %??ds milan: how is load balencing performed?


\subsection{Scalar Potential methods}
\label{sec:magstat-field-calc-pote}

This gives a formulation which can be solved using the finite element or finite difference discretisation methods. However, the zero boundary condition on $\phim$ at infinity, \eqref{eq:phi-inf}, is problematic. We obviously can not discretise an infinite domain to apply this condition since that would involve either infinite discrete elements or an infinitely sized element. Hence other techniques must be used.

The speed of the internal field calculation is given by the discretisation method, however applying the boundary conditions can require additional processing time.

% A third possibility is to define a vector potential $\vec{A}$ such that...

\subsubsection{Asymptotic Boundary Conditions}
\label{sec:asymptot-bcs}

One way to avoid an infinite domain is to truncate the external region at some finite distance from the magnetic domain. However the relationship between truncation distance and accuracy is problem dependant (since the size of the external field at any point depends on the problem geometry) and does not lead to good accuracy even for large truncation distances.

A more sophisticated method is to use asymptotic boundary conditions.\cite{Yang1997} The idea here is to use a truncated external region to calculate the boundary conditions on the magnetic domain that correspond to equation~\eqref{eq:phi-inf} being applied at infinity. Additionally the fact that any solution to the Poisson equation~\eqref{eq:nnphim} can be represented as an infinite series of harmonic functions is used to improve the accuracy. However the accuracy of this approach is still low compared to the hybrid method, even for large truncation distances.\cite{Bottauscio2008}

This method of applying the boundary conditions was used by Yang in GDM\cite{Yang1997} (general purpose dynamical micromagnetic code), but the code does not seem to be available any more.

\subsubsection{The Hybrid Boundary/Finite Element Method}
\label{sec:bound-elem-meth}

The idea of the hybrid method is to replace the external domain by a dipole layer placed on the surface of the magnetic domain which mimics the effect of the infinite external domain. This removes the need to truncate or discretise the infinite external domain. The full details of the method applied to magnetostatic calculations is discussed in \autoref{sec:hybr-finit-elem}.

A comparison by Bottauscio\cite{Bottauscio2008} found that using the hybrid method was more accurate than applying asymptotic boundary conditions for a calculation of the time evolution of the magnetisation of a sphere with zero exchange coupling. Even with a truncation distance of four times the size of the magnetic sphere (the total domain was $4^d$ times larger then the sphere) the accuracy when using asymptotic boundary conditions was worse and did not improve between truncation distances of three and four times the magnetic sphere radius. Even when exchange coupling was added (giving an easier test) the truncation method was worse than the hybrid element method.

One downside is an increase in the difficulty of creating the model since some parts of the boundary element method are mathematically different to the finite element method. For example singular integrals occur in the boundary element method and more advanced integration methods are needed. Also the hybrid method requires an additional dense matrix-vector multiplication for the calculation of boundary conditions.

The speed (and memory usage) of calculation of the boundary values in the method is limited by the dense matrix multiplication which is $\order{N_b^2}$, where $N_b$ is the number of boundary nodes. The use of hierarchical matrix techniques can reduce this to $\order{N_b \log(N_b)}$.\cite{Knittel2009} Hence the speed of the hybrid method depends on the geometry. For example in 3D structures which are roughly spherical $N_b = \order{N^{2/3}}$ which gives optimal computation speed\footnote{With hierarchical matrix techniques the speed is $\order{N^{2/3}\log(N^{2/3})}$ but $\log(x) << x^{1/2}$ for large $x$, hence $\order{N^{2/3}\log(N^{2/3})} << \order{N}$, \ie optimal computation speed scaling.} but for extremely flat structures it can be as bad as $N_b = \order{N}$.

The hybrid method was first applied to the computation of magnetostatic fields by Fredkin and Koehler.\cite{Fredkin1990}


% \section{Conclusions}
% \label{sec:model-conclusions}

% A dynamic modelling method will be used because the ability to model time dependant processes is essential.

% For the spatial discretisation a macrospin model is not considered because its application is limited to cases with well separated grains and because it is not able to model effects inside a grain. Therefore a finite element spatial discretisation will be used because the ability to treat arbitrary geometries is extremely desirable (for example in the modelling of bit patterned media). Additionally \texttt{oomph-lib}\cite{oomph-lib-website} (a multi-physics open source finite element modelling library) gives an excellent framework in which to construct a new finite element micromagnetic model. This choice will also allow easy extension to the modelling of heat assisted magnetic recording using a finite element discretisation of the heat equation to model heat flow.

% The magnetostatic field will be evaluated from a scalar potential by the hybrid finite/boundary element method. A finite element method will be used because it can be easily integrated into the overall model. The hybrid method will be used to apply boundary conditions because the accuracy loss involved in efficiently approximating the external region by asymptotic boundary conditions is large and problem dependant.

% The time discretisation scheme used will be the mid-point method because of it's preservation of magnetisation vector length and ability to deal with stiffness. The use of preconditioners to speed up to the solution of the non-linear system created at each time-step will be investigated.


%%% Local Variables:
%%% mode: latex
%%% TeX-master: "main"
%%% End:
