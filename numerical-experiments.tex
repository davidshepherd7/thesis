\chapter{Numerical experiments}
\label{cha:numer-experiments}

In this section we apply the complete FEM/BEM scheme with adaptive IMR and nodal integration, as developed in \crefs{sec:galerk-meth-llg, sec:hybr-finit-elem, sec:adaptive-imr}, to ``real-world'' micromagnetics problems.

The first problem we tackle is the \mumag standard problem 4 \cite{mumag-website}, which is widely used to test dynamic micromagnetic codes.



\section{The \mumag standard problem 4}

Widely used to test micromagnetic codes

Reversal of a thin film of permalloy under two different fields

Unfortunately FEM/BEM magnetostatic calcultations are extremely unsuited to thin film problems (the dense BEM matrix size is proportional to the size of the boundary, in thin films the entire problem is on the boundary. The additional geometric flexibility is not needed for simple cubeoid shapes.)
But we will do it anyway in order to test our results.


\subsection{Problem specification}

The problem specification is as follows:

The magnetic domain is a simple sheet of magnetic material $500 \times 125 \times 3$nm with material parameters
\begin{equation}
  \begin{aligned}
    A &= 1.3\E{-11} \text{J/m}, \\
    M_s &= 8.0\E{5} \text{A/m}, \\
    \Kone &= 0.0, \\
    \gymagc &= 2.211 \E{5} \text{m/As}, \\
    \dampc &= 0.02.
  \end{aligned}
\end{equation}
Two different applied fields should be used, correspond to two different solutions:
\begin{equation}
  \begin{aligned}
    \happ_1 = [-24.6, 4.3, 0.0] \E{-3}\text{A/m}, \\
    \happ_1 = [-35.5, -6.3, 0.0] \E{-3}\text{A/m}, \\
  \end{aligned}
  \label{eq:mumag-h-app}
\end{equation}
where we have converted from the magnetic flux intensity specified by the \mumag website to magnetic field by dropping a factor of $\mu_0$ from the RHS.
The initial condition is the result of relaxing the magnetisation from the state created by a saturating field in the $[1,1,1]$.

The magnetic parameters result in a magnetostatic exchange length (and simulation unit length) of
\begin{equation}
  l_{\text{ex}} = \sqrt{\frac{2A}{\mu_0 M_s^2}} = 5.6858\text{nm},
\end{equation}
hence the normalised dimensions are $500 \times 125 \times 3 / 5.6858$.
The unit time is
\begin{equation}
  t_{\text{unit}} = \frac{1}{\gymagc M_s} = 5.653\text{ps}.
\end{equation}
The normalised applied fields are simply the fields given in \cref{eq:mumag-h-app} divided by $M_s = 8.0\E{5}$.

\subsection{Numerical methods and parameters}

Mesh

Newton tol

quadrature(s)

time integrator(s)

solvers

etc.


\subsection{Results}

Plot dynamics with some other peoples.

Conservation properties of IMR.


\section{Reversal of a magnetic nanotube?}

More relevant problem for FEM/BEM methods--non-trivial geometry.

More relevant for conservation properties: complex, long time dynamics.


\subsection{Problem specification}

??ds


\subsection{Numerical methods and parameters}

??ds mesh generation



\subsection{Results}

??ds convergence: space and time


??ds conservation: m and energy


??ds adaptivity


??ds comparison to other methods?


%%% Local Variables:
%%% mode: latex
%%% TeX-master: "main"
%%% End:
