\chapter{Numerical experiments}
\label{cha:numer-experiments}

??ds do I use h for element size elsewhere?

??ds I've never explained how to solve eBDF3 system: pin magnetostatics, mass matrix inversion, use ll equation, with nodal quadrature we have a lumped mass matrix, no inversion needed!


In this section we apply the various numerical methods constructed throughout this thesis to some more realistic micromagnetics problems.

The first problem is a wave-like case without magnetostatics where there is a known analytical solution.
This allows us to test the convergence of the various methods when applied to the full LLG with spatial variations, as well as the conservation behaviour of the IMR.

We then show results from the \mumag standard problem \#4 \cite{mumag-website}, which is widely used to test dynamic micromagnetic codes.
This allows us to verify the complete model and to examine the performance of the various numerical methods when magnetostatic fields are included.



\section{Wave-like solution}
\label{sec:numer-exper}

The first problem is

\subsection{Problem definition}

We solve the LLG without magnetostatics on a two dimensional square domain $\magd = [0,1] \times [0,1]$ with periodic boundary conditions.
In time we simulate the solution in $[0, 2]$.

In the following experiments we use the wave exact solution in 2D (see \cref{sec:wave-like-solution}) for simplicity and because an exact solution is known.
The solution parameters used are $\kvec = 2\pi$ so that the solution is periodic on domains of unit size, $c = 0.1\pi$ which gives reasonable amplitude oscillations and $\dampc = 0.01$ due to physical relevance).
For energy conservation experiments $\dampc = 0$ is used.


This problem allows us to examine the convergence and conservation properties of IMR with the FEM and nodal quadrature.
In particular the existence of an exact solution for the LLG with exchange allows us to measure the convergence rate.
Unfortunately we do not know of any non-trivial (\ie with $\mv$ varying in both space and time) exact solution for the LLG with exchange and magnetostatics.


\subsection{Implementation details}

We use the finite element method as discussed in \cref{sec:galerk-meth-llg} to spatially discretise the LLG equation with exchange coupling (no other effective fields are included).
We use a mesh of square elements, % and a mesh of triangular elements made by splitting each square element in two with a diagonal line from $\sv = [0, 1]$ to $\sv = [1, 0]$.
unless otherwise specified we use a mesh with $5 \times 2^4$ elements along each edge.

Both the nodal quadrature discussed in \cref{sec:local-nodal-integr} and standard Gaussian quadrature are used in order to compare the two.

For time integration the IMR, TR and BDF2 methods are used, adaptive step sizes with a tolerance of $\toltt = 10^{-5}$ are used except for the convergence experiment which requires fixed step sizes.

Linearisation is handled using the Newton-Raphson method, the newton tolerance is set to $10^{-12}$ unless otherwise specified.
The linear systems are solved using GMRES with an ILU-1 preconditioner.\footnote{\hypre's Euclid preconditioner \cite{hypre} with no drop tolerance and factorisation level 1.}


\subsection{General results}


An example snapshot of the solution is shown in \cref{fig:2d-wave-snapshot}.
As time proceeds the wave moves in the $[1,1]$ direction and is simultaneous damped out to the $\mv = [0,0,1]$ state.
Time plots of the exact solution along with more details are given in \cref{sec:wave-like-solution}.

\begin{figure}
  \centering
  \includegraphics[width=0.8\textwidth]{images/placeholder}
  \caption{Snapshot of the solution}
  \label{fig:2d-wave-snapshot}
\end{figure}


??ds example time trace?

??ds These choices of mesh and time step resolve the solution well, as can be seen in the convergence experiments above.

The time step behaviour for this problem is simple: all of the integrators rapidly increase the time step to a reasonable value then keep it roughly constant (increasing slightly in the damped case).
This is as expected, in particular note that there is no oscillation of the step size with the precessional motion unlike in \cref{sec:aimr-llgode-numerical-results}.


??ds TR catastrophic failure?


\subsection{Convergence}

Since we have an exact solution for this example we can calculate the total error and plot the convergence as $\dtn \goesto 0$ and $h \goesto 0$.
Following the example of Jeong \etal \cite{Jeong2014} we link the spatial discretisation length to the time step by $\dtn = 0.32h$. (Note that in contrast to explicit time integration methods this is \emph{not} required for stability it is merely more convenient to experiment with a single parameter.)
We set $h = 1/5(2^n)$ with $n=1,2,3,4,5,6,7,8 ??ds$.

The error norm used is $\errmpde = \norm{m(\xv_j, t_n) - \mv_j,n}$. ??ds this is no good...

We plot two figures: the error after a single step and the error after some time.

??ds

??ds also look at nodal quadrature




\subsection{Conservation properties}
\label{sec:2d-wave-results-cons-prop}

% The maximum time was $t_{max} = 5$ ($\approx 20$ wave periods), the damping is small enough that the oscillations continue well past this time. % trace in folder ??ds check it
\Cref{fig:mean-ml-error-2d} shows the behaviour of the maximum (over all nodes) error in magnetisation length.
When using (adaptive) IMR with nodal quadrature the magnetisation length error remains extremely small ($\order{10^{-14}}$).
When using IMR with Gaussian quadrature or BDF2 the error rapidly grows to around $\order{10^{-4}}$ and remains there.
It is interesting to note that nodal integration has a slight beneficial effect on the magnetisation length error of BDF2.

\begin{figure}
  \centering
  \includegraphics[width=0.8\textwidth]{plots/2d_wave_solution_m_length/mlengtherrormaxesvstimes}
  \caption{Evolution of the maximum error of nodal magnetisation lengths in the 2D wave example with various time integrators and quadratures.}
  \label{fig:mean-ml-error-2d}
\end{figure}

 % ??ds rerun this?
% To check that the conservation is independent of problem parameters we ran a parameter sweep using: square and triangle elements; 36, 441 and 6561 nodes; time steps of $0.1$, $0.01$ and $0.001$; and damping parameters of $1$, $0.1$, $0.001$ and $0$.
% The maximum length error over all parameter sets, all time steps and all nodes when using nodal quadrature was 2.364775e-12, when using Gaussian quadrature it was 0.013746647.
% This clearly demonstrates the necessity and effectiveness of the nodal quadrature scheme for retaining the conservation properties of the implicit midpoint rule.
% % using the same data as for the figures above, look in their folders for parameter sets data parsing command: parse.py -d /mnt/moredata/optoomph/user_drivers/micromagnetics/experiments/parameter_sweeps/parameter_file_0/ -l=-dt -l=-damping --split=-integration --print-data max-max-ml --print-data ml -l=initial_nnode


We also examine the energy conservation properties of the various time integration schemes for the wave solution with $\dampc = 0$.
The energy is calculated using \cref{eq:nd-e-ex}.
The integrals can be evaluated exactly using any quadrature because $\grad \mv$ is a constant inside each element.

The results are shown in \cref{fig:energy-error-2d}.
??ds analysis

\begin{figure}
  \centering
  \includegraphics[width=0.8\textwidth]{plots/2d_wave_solution_energy/absofenergychangevstimes}
  \caption{Evolution of the error in energy in the undamped 2D wave example with various time integration methods, quadrature schemes and with/without re-normalisation.}
  \label{fig:energy-error-2d}
\end{figure}



\subsection{Effect of Newton tolerance}
\label{sec:effect-newt-toler-m-conservation}

Since the non-linear residual \cref{eq:weak-llg} used in the derivation of the conservation properties for energy and $\abs{\mv}$ is only true up to the accuracy of the linearisation method we would expect to see some effect when modifying this accuracy.
In our model the Newton-Raphson method is used for linearisation (see \cref{sec:newt-raph}) so the relevant measure of accuracy is the Newton tolerance, $\ntol$.

The obvious experiment to carry out would be to vary the Newton tolerance and examine how the error in $\abs{\mv}$ is affected.
However Newton's method converges extremely quickly meaning that the final residual is often orders of magnitude smaller than the tolerance, this would hide any corrolation between the tolerance and the error.
Instead we plot the error against the actual converged residual norm (specifically: the mean over time steps of $\norm{\rv}_\infty$ after the Newton method has converged).
In order to generate a variety of converged residual norms we run the experiment with a wide range of parameters: $\ntol = ??ds$, $\dtn = ??ds$, $\dampc = ??ds$ and $N_h = ??ds$.
The results are shown in \cref{fig:mean-ml-error-2d-nodal-newton-tests}, there is a clear correlation between small residuals and small $\abs{\mv}$ error.
??ds any correlation with other parameters?

A similar result is seen in \cref{} for the energy conservation property when $\dampc = 0$.

\begin{figure}
  \centering
  \includegraphics[width=0.8\textwidth]
  {plots/2d_wave_solution_m_length_newton_res/-maxmaxmathbfm-1vsmeanmathbfr_mathrmfinal_infty.pdf}
  \caption{Corrolation between maximum error of nodal magnetisation lengths and largest maximum Newton residual norm after convergence in the 2D wave example solved using adaptive IMR and nodal quadrature.}
  \label{fig:ml-error-2d-nodal-newton-tests}
\end{figure}


\begin{figure}
  \centering
  \includegraphics[width=0.8\textwidth] {images/placeholder}
  \caption{Corrolation between error in energy and largest maximum Newton residual after convergence in the undamped 2D wave example solved using adaptive IMR and nodal quadrature.}
  \label{fig:energy-error-2d-nodal-newton-tests}
\end{figure}


\subsection{Triangular meshes}

??ds sort this out?


\subsection{Conclusions}

We have shown that the conservation properties of IMR persist in a weak form FEM model when used with a nodal quadrature scheme and certain meshes.
??ds triangle mesh?




\section{The \mumag standard problem \#4}

The \mumag standard problem \#4 is the problem most commonly used to test implementations of dynamic micromagnetic models.
It involves modelling the reversal of an extremely thin cuboid film of permalloy-like material under two different applied fields.

Unfortunately FEM/BEM magnetostatic calcultations are extremely unsuited to thin film problems: the dense BEM matrix size is proportional to the number of nodes on the boundary and in a thin film every single node is on the boundary.
Additionally the key benefit of FEM/BEM (accurate resolution of complex geometries) is not required since the film is a simple cuboid.
However, since there are no other widely studied test problems and there are no problems with both magnetostatics and exchange for which an analytical solution is known, we use the standard problem to verify the model.


\subsection{Problem specification}

The problem specification is as follows:

The magnetic domain is a simple sheet of magnetic material $500 \times 125 \times 3$nm with material parameters
\begin{equation}
  \begin{aligned}
    A &= 1.3\E{-11} \text{J/m}, \\
    M_s &= 8.0\E{5} \text{A/m}, \\
    \Kone &= 0.0, \\
    \gymagc &= 2.211 \E{5} \text{m/As}, \\
    \dampc &= 0.02.
  \end{aligned}
\end{equation}
Two different applied fields should be used, correspond to two different solutions:
\begin{equation}
  \begin{aligned}
    \happ_1 = [-24.6, 4.3, 0.0] \E{-3}\text{A/m}, \\
    \happ_1 = [-35.5, -6.3, 0.0] \E{-3}\text{A/m}, \\
  \end{aligned}
  \label{eq:mumag-h-app}
\end{equation}
where we have converted from the magnetic flux intensity specified by the \mumag website to magnetic field by dropping a factor of $\mu_0$ from the RHS.
The initial condition is the result of relaxing the magnetisation from the state created by a saturating field in the $[1,1,1]$.

The magnetic parameters result in a magnetostatic exchange length (and simulation unit length) of
\begin{equation}
  l_{\text{ex}} = \sqrt{\frac{2A}{\mu_0 M_s^2}} = 5.6858\text{nm},
\end{equation}
hence the normalised dimensions are approximately $87.94 \times 21.98 \times 0.53$.
The unit time is
\begin{equation}
  t_{\text{unit}} = \frac{1}{\gymagc M_s} = 5.653\text{ps}.
\end{equation}
The normalised applied fields are simply the fields given in \cref{eq:mumag-h-app} divided by $M_s = 8.0\E{5}$.

\subsection{Implementation details}

We use the FEM to spatially discretise the LLG equation and the Newton-Raphson method to solve the resulting non-linear systems as described in \cref{cha:numer-experiments}.
The hybrid FEM/BEM method, described in \cref{sec:hybr-finit-elem}, is used for magnetostatic calculations.

The coupling of LLG equation with the magnetostatic calculations is done using both the monolithic and semi-implicit methods discussed in \cref{sec:solution-strategies}.
The solution of the monolithic linear system is done using the fully iterative preconditioner $\inexact{\precc}$, it turns out that for thin film problems the ILU-1 approximation to the LLG block, $\Fm$, is effective.
The decoupled systems are solved using the methods described in \cref{sec:llg-only-system}.

Both Gaussian quadrature and the nodal quadrature described in \cref{sec:local-nodal-integr} are used.
The TR, BDF2 and IMR adaptive time integration schemes (see \cref{sec:adapt-impl-midp,sec:aimr-implementation}) are tested.
Re-normalisation of the magnetisation is also tested for the BDF2 and TR schemes.


A structured mesh is used consisting of cuboid elements is used due to the simple geometry of the problem.
In the $z$ direction (out of the thin film plane) a single layer of elements is used at all refinements.
This is the standard approach, and is expected to give acceptable resolution because the exchange length of the material (the length scale over which the magnetisation can vary) is around twice the thickness.
The number of elements along the $x$ and $y$ axes, denoted $n_x$ and $n_y$, is varied but the ratio is fixed as $n_x = (500/125) n_y$ so that the element edge lengths in each direction are identical.
We use $n_x=75,100,125$, which gives edge lengths of 1.17, 0.89 and 0.70 exchange lengths respectively.

The Newton tolerance is fixed at $\ntol = 10^{-11}$, the adaptive integrator tolerance is $\toltt = 10^{-5}$, the initial time step is $\dtn_0 = 10^{-4}$ which is small enough to allow the adaptive integrator to naturally increase to an appropriate step size.
We cap the time step at $\dtx{\text{max}} = 4.5$ to avoid issues with linear and solver converge at extremely large time steps.


To generate the initial S-state we run the simulation starting from the state $\m=[1,1,1]$ with the applied field
\begin{equation}
  \hap(t) =
  \begin{cases}
    10 (1- \frac{t}{100}) & t < 100, \\
    0 & t \geq 100,
  \end{cases}
\end{equation}
and with $\dampc = 1.0$ for 300 time units.
The time integrator history data is then set such that it ``has been in this state forever'', and the time step size is reset to the initial value.
The relevant applied field as specified in the problem is then set and the simulation is continued.
The initial condition is always generated using the same numerical methods as are used in the simulation.


??ds mention that LL form of LLG is used for explicit step in IMR, mention CG + ... solver, diagonal mass matrix with nodal integration, .... Or move elsewhere?




\subsection{Results}


??ds turns out that non-conserving, non-renormalising causes major problems: error estimates are v. large and sticks at low step sizes or fails utterly
??ds do proper analysis of this if time?

\begin{figure}
  \centering
  \includegraphics[width=0.8\textwidth]{images/placeholder}
  \caption{Initial S-state as generated by... ??ds}
  \label{fig:intial-mumag4}
\end{figure}


\begin{figure}
  \centering
  \includegraphics[width=0.8\textwidth]{plots/mumag4_convergence/mumag4_field1-meanmxsvs-meanmysvs-meanmzsvs-dtsvstimes.pdf}
  \caption{Mean magnetisation vs time for field 1 using monolithic IMR with nodal quadrature, the legend shows the number of nodes in the problem.
    ??ds with the results given by nmag and some FD results?}
  \label{fig:nmag-comparison-mumag4-field1}
\end{figure}

\begin{figure}
  \centering
  \includegraphics[width=0.8\textwidth]{plots/mumag4_convergence/mumag4_field2-meanmxsvs-meanmysvs-meanmzsvs-dtsvstimes.pdf}
  \caption{Mean magnetisation vs time for field 2 using monolithic IMR with nodal quadrature, the legend shows the number of nodes in the problem.
    ??ds with the results given by nmag and some FD results?
  }
  \label{fig:nmag-comparison-mumag4-field2}
\end{figure}

??ds blow up of $m_y$ at time where it is dodgy?


\begin{figure}
  \centering
  \includegraphics[width=0.8\textwidth]{images/placeholder}
  \caption{The state of the system at the time when $m_x$ crosses zero}
  \label{fig:mumag4-spatial-x-crossing-0}
\end{figure}



Since the second field gives a more challenging test, for the rest of the results we only show that field.

First we plot a comparison of the magnetisation generated using IMR, nodal integration and monolithic or decoupled approach.
\Cref{fig:mumag4-spatial-x-crossing-0} shows the results, kind of different?
\begin{figure}
  \centering
  \includegraphics[width=0.8\textwidth]{images/placeholder}
  \caption{Mean magnetisation vs time for field 2 as computed by the monolithic and decoupled methods with IMR, nodal quadrature and the highest mesh resolution.}
  \label{fig:mumag4-implicit-decoupled}
\end{figure}


The maximum (over space) error in the magnetisation length vs time for decoupled and implicit methods are shown in \cref{fig:mumag4-implicit-decoupled}.
As expected both coupling approaches and all numbers of nodes conserve $\abs{\mv}$ to around the Newton tolerance.
The increase in error with more nodes is likely to be due to the increase in error accumulation because of larger numbers of calculations ??ds.
\begin{figure}
  \centering
  \includegraphics[width=0.8\textwidth]{plots/mumag4_ml/mlengtherrormaxesvstimes.pdf}
  \caption{Magnetisation length errors when using IMR with nodal integration for field 2, legend shows the number of nodes.}
  \label{fig:imr-conservation}
\end{figure}



\begin{figure}
  \centering
  \includegraphics[width=0.8\textwidth]{images/placeholder}
  \caption{Energy errors for the 0 damping case, vs time or max?}
  \label{fig:energy-conservation}
\end{figure}




Finally we examine the effectiveness of the linear solver for the monolithic method.
The iterations needed for convergence are shown in \cref{fig:mumag4-solver-iterations}, note that while they are growing with $\Nn$ they remain reasonable for the sizes required here.
The time taken to set up the preconditioner is displayed in \cref{fig:mumag4-solver-time}, again it grows with the number of nodes but remains reasonable for all cases shown here.
Also note that with nodal quadrature the preconditioner is cheaper to set up but also less effective, to counteract this a higher level of fill-in could be used with nodal quadrature.

\begin{figure}
  \centering
  \includegraphics[width=0.8\textwidth]{plots/mumag4_monolithic_its/meanofnsolveritersvsinitialnnode.pdf}
  \caption{Mean iterations to converge over all time and all newton steps all time steppers and fields, using monolithic solver with GMRES and preconditioner $\inexact{\precc}$.}
  \label{fig:mumag4-solver-iterations}
\end{figure}


\begin{figure}
  \centering
  \includegraphics[width=0.8\textwidth]{plots/mumag4_monolithic_its/meanofpreconditionersetuptimesvsinitialnnode.pdf}
  \caption{Mean time (over all newton steps and all time) to set up the preconditioner $\inexact{\precc}$ vs the number of nodes.}
  \label{fig:mumag4-solver-time}
\end{figure}


??ds characterise number of newton steps?


??ds total solve time? will be dominated by BEM multiplication... meaningless




\subsection{Conclusions}

Our method works, blah blah

FD gives different results to FEM?

Semi-implicit causes problems with TR stability, IMR energy conservation.




%%% Local Variables:
%%% mode: latex
%%% TeX-master: "main"
%%% End:
