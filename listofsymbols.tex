
%? could include section first introduced

\chapter*{List of Symbols}

% More spacing between lines in tables
\renewcommand{\arraystretch}{1.2}
% remember to reset to 1 at the end!

% A line with a title above it and small gap before and after.
\newcommand{\hlinegap}[1]{\noalign{\medskip} & \emph{#1} \\ \hline \noalign{\smallskip}}

\begin{tabular}{r  p{12cm}} %right and left aligned columns separated by a double space

  \textbf{Symbol} & \textbf{Meaning} \\
  \hline\hline %double line

  \hlinegap{Constants}
  $\mu_0$ & The magnetic constant (or the magnetic permeability of vacuum),
  %empty line to avoid line break in the middle of maths

  $\mu_0 = 4 \pi \E{-7} \approx 12.6 \E{-7} \text{ N A}^{-2}.$ \\
  $\dampc$ & Gilbert damping constant, (material-dependant). \\
  $\gymagc$ & The absolute value of the Landau-Lifshitz gyromagnetic ratio,
  %empty line to avoid line break in the middle of maths

  $\gymagc = \abs{ \frac{g_e \mu_B}{\mu_0 \hbar} } \approx 1.4 \E{17} \gymagu$. ??ds I think this might be wrong... \\

  $M_s$ & Saturation magnetisation (material-dependant). \\
  $A$ & Exchange constant (material-dependant). \\
  $K_1, K_2$ & Magnetocrystalline anisotropy constants (material-dependant). \\

  \hlinegap{Domains}
  $\magd$ & The magnetic domain \\
  $\boundd$ & The boundary between the magnetic domain and the external domain \\
  $\extd$ & The external (non-magnetic) domain \\
  $\real^d$ & $d$-dimensional Euclidean space \\

  \hlinegap{Fields and Magnetisation}
  $\Hv $ & The total effective magnetic field \\
  $\Happ$ & The applied magnetic field \\
  $\Hms$ & The magnetostatic field \\
  $\Hex$ & The effective field due to exchange coupling \\
  $\Hca$ & The effective field due to crystalline anisotropy \\
  $\Hthm$ & The effective field due to thermal effects \\
  $\Mv$ & The magnetisation \\
  $\ev$ & The (closest) magnetocrystalline anisotropy easy axis \\
  $\mv, \hv,$ etc.& Normalised magnetisation and fields \\

  \hlinegap{Magnetostatic Potentials}
  $\phim$ & The magnetostatic potential \\
  $\phi_1$, $\phi_2$ & The reduced potentials (see Section~\ref{sec:problem-description}) \\
  $\phi^\inte_i,\phi^\exte_i$ & The value of $\phi_i$ immediately inside/outside the magnetic domain (for $i = 1,2,m$) \\
  $\Green$ & The Green's function for the Poisson equation \\
  $\gamma = \frac{\alpha}{\alpha_{\text{max}}}$ & The fractional angle/solid angle (in 2D/3D respectively) \\

  \hlinegap{Finite Element Method}
  $\tbf$ & Test function (also the finite element basis function -- see Section~\ref{sub:Actual-Finite-Elements}) \\
  $n$ & Node index (equivalent to test function index in finite elements) \\
  $l$ & Shape function index \\
  $e$ & Element index \\

  \noalign{\smallskip}\hline
\end{tabular}


%%% Local Variables:
%%% mode: latex
%%% TeX-master: "main"
%%% End:
