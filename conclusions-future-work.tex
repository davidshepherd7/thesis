\chapter{Conclusions and future work}
\chaptermark{Conclusions}

\section{Conclusions}

Introduced a complete FEM + NR + implicit time integration + FEM/BEM + linear solvers model.
Validated on analytical solution
Sort of validated on \mumag standard problem \#4

Found an effective preconditioner for the monolithically coupled LLG with FEM/BEM magnetostatics problem.
Such coupling methods should be useful in getting energy conservation of IMR
Also potential benefits over semi-implicit coupling for stochastic problems, stability.


Invented generally applicable, efficient and effective adaptive IMR
Shown that the adaptivity works well for a wide variety of ODE test cases
Shown that the geometrical integration properties carry through in the ODE and PDE (without magnetostatics) cases

Studied the performance of common implicit time integration schemes on some exact solutions.
Found that BDF2 always worse than TR and IMR
But analytical solutions too simple for analysis of geometric integration effects

Studied effect of discretisation length on relative performance of implicit and explicit time integration methods (aka stiffness).
Showed that stiffness in micromagnetics can arise from discretisation alone, more stiff for smaller edge lengths, as expected from classic PDE theory.
Found that for simple problems explicit methods are sufficient, but there are many classes of problem where implicit methods are expected to be more efficient.
Especially when FEM/BEM is introduced.


Had some issues...
Non-conservation of energy with magnetostatics -- almost certainly due to collocation BEM which results in a non-symmetric discrete operator.
Non-conservation of magnetisation length on triangular meshes -- no idea where this could come from, need to test with other peoples code.
Errors in field 2 of mumag, probably due to FEM/BEM difficulties with corner singularities

\section{Future work}

Simple optimisations: store linear blocks of Jacobian between solves: $19/25$ of the $N \times N$ blocks are linear or empty, so this should speed up assembly significantly (the 3 diagonal LLG blocks are mass matrices, all magnetostatics blocks are linear or empty).
Similarly almost all contributions to the 6 non-linear blocks are skew symmetric and so can be reused.
With these optimisations Jacobian assembly time (which makes up around half of the runtime for time steps where an iterative can be used) could be reduced by around a factor of 10.
Also using biCGStab or restarted GMRES could reduce computation time when a large number of iterations are needed (although both have reduced convergence guarantees, hence why GMRES was used so far).


Integrate the linear solvers with an efficient and robust multigrid based preconditioner for the Newton-Raphson linearised LLG equation.\footnote{The multigrid smoother developed in \cite{Jeong2014} is probably not relevant because it relies on a cancelation only present in finite-difference simulations.}
Also integrate hierarchical BEM matrix.
This will allow linear scaling (\ie $\order{N}$) in both space and time independently for most geometries ($\order{N \log N}$ for surface dominated geometries), as discussed in \cref{sec:desir-prop-numer}.


With such an enhanced solver the effectiveness of the preconditioner discussed in \cref{sec:solution-strategies} on extremely large problems could be investigated.
It is possible that neglecting the entire BEM block could cause problems in cases which are beyond the reach of the generic ilu preconditioner used here.


Investigate the use of Galerkin BEM in magnetostatic calculations, the \hlib developers consider Galerkin to be superior to collocation methods but (as far as I am aware) all existing micromagnetics models use a collocation approach.
Also Galerkin BEM methods should restore the energy conservation properties of IMR.


Investigate the effects of the various geometric integration properties of IMR on the overall accuracy for a variety of problems.
In particular: does loss of symplectic properties due to adaptive time integration have a noticeable impact on the accuracy?


Investigate the relationship between ``no spurious damping'' and energy conservation with $\dampc = 0$.
I suspect energy conservation is a result of ``no spurious damping'' and length conservation combined, this could be tested using TR with Lagrange multipliers for length conservation (see \eg \cite{Szambolics2008a}).


%%% Local Variables:
%%% mode: latex
%%% TeX-master: "main"
%%% End:
