\chapter{Conclusions and future work}

\section{Conclusions}

blah blah

??ds


\section{Future work}

Simple optimisations: store linear blocks of Jacobian between solves: $19/25$ of the $N \times N$ blocks are linear or empty, so this should speed up assembly significantly (the 3 diagonal LLG blocks are mass matrices, all magnetostatics blocks are linear or empty).
Similarly almost all contributions to the 6 non-linear blocks are skew symmetric and so can be reused.
With these optimisations Jacobian assembly time (which makes up around half of the runtime for time steps where an iterative can be used) could be reduced by around a factor of 10.
Also using biCGStab or restarted GMRES could reduce computation time when a large number of iterations are needed (although both have reduced convergence guarantees, hence why GMRES was used so far).


Integrate the linear solvers with an efficient and robust multigrid based preconditioner for the Newton-Raphson linearised LLG equation.\footnote{The multigrid smoother developed in \cite{Jeong2014} is probably not relevant because it relies on a cancelation only present in finite-difference simulations.}
Also integrate hierarchical BEM matrix.
This will allow linear scaling (\ie $\order{N}$) in both space and time independently for most geometries ($\order{N \log N}$ for surface dominated geometries), as discussed in \cref{sec:desir-prop-numer}.


With such an enhanced solver the effectiveness of the preconditioner discussed in \cref{sec:solution-strategies} on extremely large problems could be investigated.
It is possible that neglecting the entire BEM block could cause problems in cases which are beyond the reach of the generic ilu preconditioner used here.


Investigate the use of Galerkin BEM in magnetostatic calculations, the \hlib developers consider Galerkin to be superior to co-location methods but (as far as I am aware) all existing micromagnetics models use a co-location approach.


Investigate the effects of the various geometric integration properties of IMR on the overall accuracy for a variety of problems.
In particular: does loss of symplectic properties due to adaptive time integration have a noticeable impact on the accuracy?


Investigate the relationship between ``no spurious damping'' and energy conservation with $\dampc = 0$.
I suspect energy conservation is a result of ``no spurious damping'' and length conservation combined, this could be tested using TR with Lagrange multipliers for length conservation (see \eg \cite{Szambolics2008a}).


%%% Local Variables:
%%% mode: latex
%%% TeX-master: "main"
%%% End:
