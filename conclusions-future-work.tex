\chapter{Conclusions and future work}
\chaptermark{Conclusions}

\section{Conclusions}

In this thesis we have introduced a complete numerical method for the solution of dynamic micromagnetic problems using the finite element method with Newton-Raphson linearisation, a variety of implicit time integration schemes, hybrid FEM/BEM magnetostatics calculations and efficient iterative linear solvers.
The methods have been validated against an analytical solution for a problem without magnetostatics and against the \mumag standard problem \#4 with magnetostatics.

In \cref{sec:adaptive-imr} we have demonstrated a novel and widely applicable adaptive time step selection algorithm for the implicit midpoint rule.
We have also shown that the time step selection works well for a wide variety of ODE test cases, as well as a number of PDE test cases using the LLG equation.
Additionally we have shown that the geometrical integration properties of the constant time step IMR are also applicable to the adaptive time step version.

In \cref{sec:solut-coupl-syst,sec:numer-exper-fem-bem-systems,sec:mumag-stand-probl} we introduced and tested efficient, robust and scalable solution methods for the monolithically coupled LLG-FEM/BEM magnetostatics problem provided that a good preconditioner for the LLG system alone is available.
Such monolithic couplings are required to obtain the energy property of the implicit midpoint rule and may offer advantages in stochastic integration methods.

In \cref{sec:imr-ode-llg-numer-exper,sec:numer-exper} we studied the performance of common implicit time integration schemes on some micromagnetic problems with exact solutions.
We found that the BDF2 scheme always performs poorly compared to the TR and IMR schemes, this is probably due to a combination of the larger local truncation error and the spurious numerical damping of BDF2.
% Unfortunately we were unable to analyse the effect of geometric integration on the error accumulation due to the fact that TR also displays some geometrical integration properties for these simple examples.

Finally in \cref{cha:stiffn-llg-equat} we studied the effect of discretisation on the relative performance of implicit and explicit time integration schemes (stiffness).
We found that stiffness in micromagnetics can arise from discretisation alone, and that the stiffness increases as the element size is decreased, as expected from standard PDE theory.



\section{Future work}


In the course of our numerical experiments we uncovered some issues with the geometrical integration properties of our complete model.
Firstly we found that the magnetisation length conservation property of IMR with FEM and nodal quadrature did not work when applied to meshes of triangular elements, at least in our implementation.
This effect is fairly small and is not contradicted by any numerical results in the literature that we are aware of due to the common use of comparatively loose linearisation tolerances.
An alternative implementation of IMR with FEM and nodal quadrature should be used with a tight linearisation tolerance in order to find out if this effect is an artefact of our implementation or a real issue.

Secondly we found that when FEM/BEM magnetostatics calculations discretised by a collocation based approach were included the energy conservation property of IMR is lost.
This is probably due to the asymmetry of the discrete BEM operator, which could be corrected by the use of alternative formulations of the method.
In particular the use of the Garc\'{i}a-Cervera-Roma formulation \cite{Garcia-Cervera2006} with a Galerkin discretisation approach \cite[75]{Wrobel2002} should remove this issue.

Once these issues have been resolved the relative performance of schemes with geometric integration properties should be evaluated for realistic problems.

In the area of linear solvers, an efficient, robust and scalable preconditioner for the Newton-Raphson linearised LLG equation is still required.
One approach to the construction of such a preconditioner could be to exploit the block structure of the Jacobian matrix and to use multigrid-based methods to approximate the Laplacian-like skew-symmetric blocks resulting from the exchange effective field.
With such an enhanced LLG preconditioner the effectiveness of the preconditioner discussed in \cref{sec:solution-strategies} on extremely large problems could be investigated.
Another approach to such a preconditioner for the case of granular or patterned media could be to use a domain decomposition preconditioner.
In such a method the small matrix block associated with each grain/island would be inverted by a direct solver and used as a block diagonal preconditioner.
Due to the weak coupling this would provide a good approximation for the inverse of the entire LLG block.

In the time integration of the stochastic LLG (as used for non-zero temperature models)
only a few time integration schemes are known to converge to the correct solution, one of which is the implicit midpoint rule.
Since semi-implicit magnetostatics coupling means no-longer using IMR itself, a monolithic coupling scheme is required to maintain this property.
The preconditioner developed in \cref{sec:solution-strategies} should be tested in this capacity.



%%% Local Variables:
%%% mode: latex
%%% TeX-master: "main"
%%% End:
